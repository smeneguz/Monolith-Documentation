\documentclass[10 pt,a4paper, openany]{article}
%titlepage
\usepackage[hidelinks]{hyperref}
\usepackage[italian]{babel}
\usepackage[T1]{fontenc}
\usepackage[utf8x]{inputenc}
\usepackage{amsfonts}
\usepackage{multicol}
\usepackage{graphicx}
\usepackage{amsmath}
\usepackage{framed}
\usepackage{extarrows}
\usepackage{cancel}
\usepackage{eurosym}
\usepackage{listingsutf8}
\usepackage{lastpage}
\usepackage{rotating} 
\usepackage{multirow}

\usepackage{caption}
\usepackage{makecell}
\usepackage{longtable}
\usepackage{array}
\date{}


\usepackage{makeidx}
\makeindex
\usepackage{fancyhdr}
\pagestyle{fancy}
\lhead{\includegraphics[width=.6cm]{../../../../../file_comuni/immagini/obelisk_sample_02.png}
  Obelix Group}
\chead{}
\rhead{\rightmark }% \leftmark}%da rimettere
\lfoot{}
\cfoot{}
\rfoot{\thepage / \pageref*{LastPage}}
%%%%%%%%%%%%%%%%%%%%%%%%%%%%%%%%%%%%%%
%\usepackage{lipsum}
\usepackage{../../../../../file_comuni/copertina2}
\nomedoc{Verbale interno 2017-04-20}
\versione{v1\_0\_0}
\datacreazione{2017/04/20}
\verifica{Riccardo Saggese}
\approvazione{Nicolò Rigato}
\redazione{Emanuele Crespan \eanche Federica Schifano}
\uso{interno}
\distribuzione{Prof. Tullio Vardanega \eanche Prof. Riccardo Cardin \eanche Red Babel \eanche Gruppo Obelix}
\sommario{Verbale dell'incontro tra i membri del gruppo \emph{Obelix} in data 2017-04-20}


\begin{document}
\paginatitolo
\section{Informazioni sulla riunione}

\begin{itemize}
\item[] Data: 2017-04-20
\item[] Luogo: Torre Archimede - Via Trieste, 63 Padova
\item[] Ora: 16:00
\item[] Durata: 60'
\item[] Partecipanti interni: Obelix
  \begin{itemize}
  \item[] Emanuele Crespan
  \item[] Federica Schifano
  \item[] Nicolò Rigato
  \item[] Riccardo Saggese
  \item[] Silvio Meneguzzo
  \item[] Tomas Mali
 \end{itemize}
\end{itemize}

\section{Argomenti trattati}
\begin{enumerate}
	\item Discussione in merito alla progettazione dell’SDK 
	\item Suddivisione del lavoro
	\item Correzioni documenti RR
\end{enumerate}

\section{Decisioni Prese}
\begin{enumerate}
	\item Per la progettazione si è deciso che:
	Saranno presenti delle funzionalità che una volta impostati "i file iniziali" (scaricati i file) per la creazione della bolla (inizialmente vuota), tramite queste funzionalità si potrà inserire vari widget o altri contenitori dove a loro volta si potrà modificare ulteriormente.

	\item Il lavoro è stato suddiviso in maniera equa tra tutti i membri ed ognuno lavorerà su parti di documenti distinte in modo da evitare possibili conflitti.	

	\item I documenti precedentemente segnalati come "da rivedere", verranno corretti suddividendo il lavoro tra tutti i membri del gruppo, non interferendo così in contemporanea sullo stesso documento.
	
	
\end{enumerate}


\end{document}
