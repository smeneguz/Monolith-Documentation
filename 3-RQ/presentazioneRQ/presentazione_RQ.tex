%\documentclass{beamer}
\documentclass[RawSienna,dvipsnames]{beamer}

%%% Dichiarazione dei pacchetti standard.
\usepackage[italian]{babel}
\usepackage[utf8x]{inputenc}
\usepackage{eurosym}
%%% Personalizzazione del layout---articolata su cinque livelli.
%\usetheme{split}        % layout complessivo. 
%\usetheme{CambridgeUS}
\usetheme{AnnArbor}
\useinnertheme{rectangles} % layout interno.
\useoutertheme{infolines} % layout esterno.
\setbeamercolor{title}{fg=RawSienna}
\setbeamercolor{frametitle}{fg=RawSienna}
\usecolortheme{crane} % schema di colori.
\usefonttheme{professionalfonts}  % schema dei font.
\setbeamercolor{item}{fg=RawSienna}
% Inutile dire che se volete tutti i default, potete risparmiarvi gli ultimi
% quattro comandi. 
%%% Titolo e autore.
\title{Monolith}
\subtitle{An interactive bubble provider}
\author{Revisione di Qualifica}
%\institute{Gruppo Utilizzatoti Italiani di \TeX}
%\date{\today}
\date{29 agosto 2017}

\begin{document}
	
\begin{frame}
	\begin{center}
		\includegraphics[scale=0.13]{img/obelix.png}
		\qquad\qquad
		\includegraphics[scale=0.13]{img/monolith.png}
	\end{center}
	\titlepage
\end{frame}


\section[Sommario]{}
\begin{frame}
	\tableofcontents
\end{frame}	

\section{Il gruppo Obelix}
\begin{frame}
	
	\begin{columns}
		\begin{column}{0.2\textwidth}
			
		\end{column}
		
		\begin{column}{0.4\textwidth}
			\begin{itemize}
				\item Emanuele Crespan
				\item Tomas Mali
				\item Silvio Meneguzzo
				\item Nicolò Rigato
				\item Riccardo Saggese
				\item Federica Schifano
			\end{itemize}
		\end{column}
		
		\begin{column}{0.2\textwidth}
			
		\end{column}
	\end{columns}
	
	
\end{frame}

\section{Scopo del progetto}
\begin{frame}
	\frametitle{Scopo del progetto}
	\begin{center}
	\begin{large}
		\textbf{Monolith}: Framework per la creazione di bolle interattive \\
		\vspace{1cm}
		\textbf{Demo}: Bolla esempio che usa Monolith
	\end{large}
	\end{center}
	
\end{frame}

\section{Copertura dei Requisiti}
\begin{frame}
	\frametitle{Copertura dei Requisiti}
	\includegraphics[scale=0.50]{img/Requisiti.png}
	
\end{frame}

\section{Nel prossimo periodo...}
\begin{frame}
	\frametitle{Nel prossimo periodo...}
	
	\begin{columns}
		\begin{column}{0.4\textwidth}
			
		\end{column}
		
		\begin{column}{0.5\textwidth}
			\begin{itemize}
				\begin{Large}
				\item Codice Demo
				\vspace{0.5cm}
				\item Manuali
				\vspace{0.5cm}
				\item Test
				\end{Large}		
			\end{itemize}
		\end{column}
		
		\begin{column}{0.2\textwidth}
			
		\end{column}
	\end{columns}
	
\end{frame}

\section{Test}

\subsection{Stato dei Test}
\begin{frame}
	\frametitle{Stato dei Test}
	\begin{center}
	\includegraphics[scale=0.55]{img/TU.png}
	\end{center}
\end{frame}

\begin{frame}
	\frametitle{Stato dei Test}
	\begin{center}
	\includegraphics[scale=0.55]{img/TI.png}
    \end{center}
\end{frame}

\begin{frame}
	\frametitle{Stato dei Test}
	\begin{center}
	\includegraphics[scale=0.55]{img/TS.png}
    \end{center}
\end{frame}

\subsection{Strumenti usati}
\begin{frame}
	\frametitle{Strumenti usati}
	\begin{center}
	\includegraphics[scale=0.30]{img/strumenti.png}
	\end{center}
\end{frame}
 %fede e tomas
\section{Descrizione prodotto}


\subsection{Monolith SDK}


\begin{frame}
  \frametitle{Funzionamento SDK}
  \begin{center}
    \includegraphics[width=\linewidth,height=.8\textheight,keepaspectratio]{img/uso_sdk.png}
  \end{center}
\end{frame}

\begin{frame}
  \begin{columns}[T] % align columns
    \begin{column}{.48\textwidth}
      \textbf{Creazione GUI}
      \begin{itemize}
      \item Bubble
        \begin{itemize}
        \item Sender
        \item Receiver
        \end{itemize}
        \pause
      \item Configuration Menu
      \item Button
        \begin{itemize}\pause
        \item[$ \rightarrow $] Personalizzazione
        \end{itemize}
      \end{itemize}
    \end{column}
    \pause
    \begin{column}{.48\textwidth}
      \textbf{Server Side Operations}
      \begin{itemize}
      \item[] Meteor.method
        \begin{itemize}
        \item insert  $ \rightarrow $ Custom method
        \item update $ \rightarrow $ Custom method
        \item remove
        \end{itemize}
      \end{itemize}
    \end{column}

  \end{columns}

\end{frame}

\subsection{Interfaccia}
\begin{frame}
	\frametitle{Interfaccia}
	\begin{center}
	\includegraphics[scale=0.22]{img/interf.png}
	\end{center}
\end{frame}


\begin{frame}
	\frametitle{CSS}
	\begin{center}
	\includegraphics[scale=0.35]{img/css.png}
	\end{center}
\end{frame}

\subsection{Demo}
\begin{frame}
	\frametitle{Demo: lista}
	\begin{center}
	\includegraphics[scale=0.25]{img/demo.png}
	\end{center}
\end{frame}

\section{Poblematiche riscontrate e soluzioni adottate}
\subsection{Generali}

\begin{frame}
	\frametitle{Poblematiche riscontrate e Soluzioni adottate}
	
		
		\begin{columns}
			\begin{column}{0.2\textwidth}
				
			\end{column}
			
			\begin{column}{0.4\textwidth}
				\begin{itemize}
					\item Database
					\item User Interface
					\item Test
					\item Documentazione
				
				\end{itemize}
			\end{column}
			
			\begin{column}{0.2\textwidth}
				
			\end{column}
		\end{columns}
		
		
\end{frame}





\section{Preventivo e Consuntivo}
%\subsection{Progettazione Architetturale}
\begin{frame}
  \frametitle{Progettazione Architetturale}
  \begin{center}
  	\includegraphics[scale=0.5]{img/prevPA}
  	\includegraphics[scale=0.4]{img/cakePA}
  \end{center}
Rispetto al preventivo, sono emerse differenze con una diminuzione di 1 ora sui ruoli di Responsabile, Amministratore e Verificatore, ed una maggiorazione di 3 ore per quello del Progettista.
\end{frame}

\begin{frame}
	\frametitle{Progettazione di Dettaglio}
	\begin{center}
		\includegraphics[scale=0.5]{img/prevPD}
		\includegraphics[scale=0.4]{img/cakePD}
	\end{center}
	Rispetto al preventivo, sono emerse differenze con una diminuzione di 1 ora sul ruolo di Amministratore  ed una maggiorazione sui ruoli di  Amministratore e Progettista.
	
\end{frame}
\end{document}
