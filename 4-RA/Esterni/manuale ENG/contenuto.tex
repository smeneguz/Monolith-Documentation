\section{Introduction}
\subsection{Intro}
Monolith is an SDK to build interactive bubbles in Rocket.Chat environment.
The following document explains how to install and use Monolith SDK.
It provides code snippet and examples on how to create a custom bubble on Rocket.chat using the SDK.
There are 4 repositories containing relevant code:
\begin{itemize}
\item \textbf{monolith-sdk}\\
It contains the code for the SDK. Here core funtionalities are provided. All other packages depends on this.\\
URL\\
This package includes some built-in bubbles:
\begin{itemize}
\item \textbf{CurrencyBubble}\\
This bubble allows to convert monetary values from one currency to another.
\item \textbf{PollBubble}\\
This bubble allows to make a survey.
\item \textbf{RandomBubble}\\
This bubble allow to roll a dice.
\end{itemize}

\item \textbf{monolith-hello}\\
It contains the code for the simplest example possible of how to use Monolith-SDK.\\
URL
\item \textbf{monolith-demo}\\
It contains the code for the official Monolith Demo. It represents a more complex use case than monolith-hello. See description for details.\\
URL
\item \textbf{monolith-emptybubble}\\
It contains the code for the empty bubble i.e. a set of files you can use to start to build your own bubble. See the developer guide for details.\\
URL
\end{itemize}

\subsection{How to install monolith-sdk}
To install Monolith SDK on Rocket.chat you need to follow the following steps:

\begin{lstlisting}
    git clone https://github.com/RocketChat/Rocket.Chat.git
    cd Rocket.Chat
    meteor npm start # when finished close meteor
    meteor add templating blaze-html-templates
    		react-meteor-data maxharris9:classnames
    		react-template-helper	
    meteor npm i react react-dom bluebird simpl-schema
    		react-addons-pure-render-mixin money
    		request request-promise  --save
    # copy monolith-sdk folder inside
    # the packages folder on Rocket.chat
    meteor add monolith-sdk
    meteor
\end{lstlisting}

\subsection{How to install monolith-hello}
\textbf{Precondition:} monolith-sdk must be already installed as seen in the previous section.

\begin{lstlisting}
# download from the repository monolith-hello
# and put the directory in Rocket.Chat/packages
# execute in the main directory of RocketChat:
meteor add monolith-hello
\end{lstlisting}

\subsection{How to install monolith-demo}
\textbf{Precondition:} monolith-sdk must be already installed as seen in the previous section.

\begin{lstlisting}
# download from the repository monolith-demo 
# and put the directory in Rocket.Chat/packages
# execute in the main directory of RocketChat:
meteor add monolith-demo
\end{lstlisting}




\subsection{Error and bug reporting}
If you encounter any error or bug during the normal use of the
application, please report it to us by opening an issue on our GitHub
repository at the following URL:
\url{https://github.com/ObelixSWE/monolith-sdk/issues}. \\ 
We will work to continuously improve our application and fix all the bugs and error that will be reported. 

% ------------------------------------------- 
\section{New bubble creation}
\subsection{A new bubble from an empty bubble}
monolith-emptybubble is provided as a starting point to develop a new bubble.
The first thing to do is to choose a name for your bubble. From now will be used
 <yourbubble> to indicate your project.
Once you downloaded the directory modify files as follow:
\begin{itemize}
\item rename the main directory and all files to a different name such it includes <yourbubble>
\item \textbf{package.js}
	\begin{itemize}
	\item change package name to <yourbubble> in Package.describe name.
	\item substitute empty with <yourbubble> in api.addFiles
	\end{itemize}
	\item \textbf{client/main.js}: correct file path
	\item \textbf{server/main.js}: correct file path
	\item \textbf{server/Methods.js}: write your own Meteor.method. It is required for update but is optional for insertion. Usage is explained in section \emph{Database configuration and possible uses}
	\item \textbf{lib/emptyDb.js}: rename the collection using <yourbubble>
	\item \textbf{lib/emptyCheck.js}: rename the check object and insert your data schema as described in \emph{Data schema description}
	\item \textbf{lib/emptyBubble.jsx}: 
	Gui component. See \emph{Gui requirement} for details.
	\item \textbf{lib/emptyBubbleConfig.jsx}: 
	Gui component. See \emph{Gui requirement} for details.
	\item \textbf{lib/emptyBubbleCreationButton.jsx}: 
	Gui component. See \emph{Gui requirement} for details.
	\item \textbf{lib/emptyCreator.jsx}: substitute empty for
     <yourbubble>. This file is required to integrate your graphical
     elements with monolith's gui.
\end{itemize}

Keep in mind that names you gave here need to precise. Monolith relies
on that.


\subsection{Gui requirements}
These are the gui components you must realize to obtain a bubble. You can ease your work using the component library provided with Monolith.
\subsubsection{emptyBubble.jsx}
You can realize up to 2 versione of this, for sender and for receiver,
or just use one as in examples. Just modify accordingly
<yourbubble>Creator.js.

You have to realize a React component that will receive in props all
database data for that bubble.
If you need to modify bubble server status you'll have to use
<yourbubble>Db.



\subsubsection{emptyBubbleConfig.jsx}
You have to realize a React component that will receive in props a
\emph{closeMenu} function. Use this to close menu when finished to
configure your bubble like this:

\begin{lstlisting}
    send(){
        let insProm = <yourbubble>Db.insert({...});
        insProm.then(
            (result) => {this.props.closeMenu();},
            (error) => {console.log(error);}
        );
    }
\end{lstlisting}

The example uses the promise syntax. If operation succeedes the menu
is closed.

To insert a bubble into server side database you'll have to use
<yourbubble>Db.

\subsubsection{emptyBubbleCreationButton.jsx}
This is the button in the menu to start to configure <yourbubble>.
For a basic usage you can simply insert <yourbubble> in the given
file.

If you need two button (like monolith-demo) you can redefine the
method \emph{secondAreaName} (commented in empty). You also have to
pass as props \emph{secondButtonName}.
Note that you will need a second configmenu and creator named accordingly.


\subsection{Data schema description}
In <yourbubble>Check.js you have to define a schema to validate data you put in
database.
Define it with SimpleSchema syntax:
\url{https://www.npmjs.com/package/simpl-schema}

\subsection{Database configuration and possible uses}

In <yourbubble>Db.js you simply have to insert the correct name for your
bubble.
In <yourbubble>Methods.js you can insert you can put Meteor methods to
be called at insertion or update of bubble:
\paragraph{Method for insertion} \emph{Use is optional.} 

\begin{lstlisting}
newDataObj method(dataObj);
\end{lstlisting}

This alters the data Object sent from client before it is
checked. Note the modified object must be returned.
Call it by inserting the method's name as argument of <yourbubble>Db.

\paragraph{Method for update} \emph{Use is mandatory.} 

\begin{lstlisting}
boolean method(bubbleId, argument);
\end{lstlisting}

In this method you have to put all database update operation for
<yourbubble>. It must return true for success, false for failure.
Call it by inserting the method's name as argument of <yourbubble>Db.



% ------------------------------------------- 

\section{Sdk's Classes Description}
This section explains how to use the Monolith library classes.

\subsection{SingleComponents}
\begin{flushleft}
Singlecomponent classes represent the components that can be rendered.

\paragraph{CheckButton}
CheckButton represent a HTML <checkbox> tag.
\begin{verbatim}
<CheckButton
    id="HTML id"
    classes="CSS classes"
    getCheck={this.functionName}
    value="checkbox value"
/>
\end{verbatim}
getCheck is a props that hold a function called when the checkbox onChange event is called and passes to the function a variable containing the state of the checkbox.
\begin{verbatim}
functionName(m){...}

m={id:"id", value:"value", check:[true/false]};
\end{verbatim}

    \paragraph{CheckBoxList}
CheckBoxList represent a group of CheckButton.
CheckBoxList needs to have an array passed like this:
\begin{verbatim}
let opt=[{id: 1, value: 'Hello World'},{id: 2, value: 'Installation'}];

<CheckBoxList
    classes="CSS classes"
    options={opt}
    getCheck={this.functionName}
/>
\end{verbatim}
getCheck like the CheckButton getCheck.
\begin{verbatim}
functionName(m){...}

m={id:"id", value:"value", check:[true/false]};
\end{verbatim}

    \paragraph{ComboBox}
ComboBox represent a HTML <select> tag.
\begin{verbatim}
<ComboBox
    id="HTML id"
    classes="CSS classes"
    options={["a","b","c"]}
    getSelection={this.functionName}
/>
\end{verbatim}
getSelection is a props that hold a function called when the select onChange event is called and passes to the function a variable containing the selected option.
\begin{verbatim}
functionName(m){...}

m="selected option";
\end{verbatim}

    \paragraph{Image}
Image represent a HTML <img> tag.

\begin{verbatim}
<Image
    id="HTML id"
    classes="CSS classes"
    src="img source location"
    alt="image description"
    width="image width"
    height="image height"
/>
\end{verbatim}

    \paragraph{ImageButton}
ImageButton represent a button with an image.
\begin{verbatim}
<ImageButton
    id="HTML id"
    src="img source location"
    alt="image description"
    width="image width"
    height="image height"
    handleClick={this.functionName}
/>
\end{verbatim}
handleClick is a props that hold a function called when the ImageButton is clicked.
\begin{verbatim}
functionName(id){...}
id="id of the clicked button"
\end{verbatim}
    \paragraph{LineEdit}
LineEdit represent a HTML text <input> tag.
\begin{verbatim}
<LineEdit
    id="HTML id"
    classes="CSS classes"
    updateState={this.functionName}
    value="default value"
 />
\end{verbatim}
updateState is a props that hold a function called when onChange event of the text input is called.
\begin{verbatim}
updateState(text,id){...}

text="text of the text input"
id="id of the text input"
\end{verbatim}

    \paragraph{LineEditComboBox}
LineEditComboBox represent a HTML text <input> and a HTML <select> tag.
\begin{verbatim}
<LineEditComboBox
    idle="LineEdit HTML id"
    idcb="ComboBox  HTML id"
    classesle="LineEdit CSS classes"
    classescb="ComboBox CSS classes"
    textUpdate={this.functionName1}
    options={["a","b","c"]}
    comboUpdate={this.functionName2}
/>
\end{verbatim}
textUpdate is a props that hold a function passed to the LineEdit.
comboUpdate is a props that hold a function passed to le ComboBox.
\begin{verbatim}
functionName1(text,id){...}

text="text of the text input"
id="id of the text input"

functionName2(m){...}

m="selected option";
\end{verbatim}
    \paragraph{PushButton}
PushButton represent a HTML <button>.
\begin{verbatim}
<PushButton
    id="HTML id"
    classes="CSS classes"
    handleClick={this.functionName}
    buttonName="button name"
/>
\end{verbatim}
handleClick is a props that hold a function called when the button is clicked.
\begin{verbatim}
functionName(id){...}

id="id of the button"
\end{verbatim}

    \paragraph{LineEditPushButton}
LineEditPushButton represent LineEdit and a PushButton.
\begin{verbatim}
<LineEditPushButton
    idle="LineEdit HTML id"
    idpb="PushButton HTML id"
    classesle="LineEdit CSS classes"
    classespb="PushButton CSS classes"
    getText={this.functionName}
    buttonName="button name"
/>
\end{verbatim}
getText is a props that hold a function called when the PushButton is clicked and passes to the function a variable containing the text of the LineEdit.
\begin{verbatim}
functionName(text){...}

text="text of the LineEdit"
\end{verbatim}

    \paragraph{RadioButtonGroup}
RadioButtonGroup represent a group of HTML radio button <input>.
\begin{verbatim}
<RadioButtonGroup
    classes="CSS classes"
    options={["a","b","c"]}
    getValue={this.functionName}
/>
\end{verbatim}
getValue is a props that hold a function called when a radio button is clicked and passes to the function a variable containing the selected radio information.
\begin{verbatim}
functionName(value){...}

value="value of the selected radio button"
\end{verbatim}

    \paragraph{TextAreaButton}
TextAreaButton represent a HTML <textarea> tag and a PushButton.
\begin{verbatim}
<TextAreaButton
    idta="textArea HTML id"
    classesta="textArea CSS classes"
    idpb="PushButton HTML id"
    classespb="PushButton CSS classes"
    getText={this.functionName}
    width="textarea width"
    height="textarea height"
    buttonName="button name"
/>
\end{verbatim}
getText is a props that hold a function called when the PushButton is clicked and passes to the function a variable containing the text of the textarea.
\begin{verbatim}
functionName(text){...}

text="text of the textarea"
\end{verbatim}

    \paragraph{TextAreaComboBox}
TextAreaComboBox represent a HTML <textarea> tag and a ComboBox.
\begin{verbatim}
<TextAreaComboBox
    idtx="textArea HTML id"
    classestx="textArea CSS classes"
    idcb="combobox HTML id"
    classescb="combobox CSS classes"
    width="textarea width"
    height="textarea height"
    textUpdate={this.functionName1}
    options={["a","b","c"]}
    comboUpdate={this.functionName2}
/>
\end{verbatim}
textUpdate is a props that hold a function called when onChange event of the textarea is called.\\
comboUpdate is a props that hold a function called when the select onChange event is called and passes to the function a variable containing the selected option.
\begin{verbatim}
functionName1(text){...}

text="text of the textarea"

functionName2(m){...}

m="selected option";
\end{verbatim}
\end{flushleft}

\subsection{Layout}
\begin{flushleft}
Layout are classes that represent are containers that place the elements conteined in a certain way.

\paragraph{VerticalLayout}
VerticalLayout represent a container that place the elements contained in a vertically.

\begin{verbatim}
<VerticalLayout hide={"visibility state(true or false)"}>
    <Children/>
    <Children/>
    .
    .
    .
</VerticalLayout>
\end{verbatim}

\paragraph{HorizontalLayout}
HorizontalLayout represent a container that place the elements contained in a horizontally.\\
The maximum number of element that can be displayed horizontally is 12
due to Bootstrap limits.

\begin{verbatim}
<HorizontalLayout hide={"visibility state(true or false)"}>
    <Children/>
    <Children/>
    .
    .
    .
</HorizontalLayout>
\end{verbatim}
\end{flushleft}


% ------------------------------------------- 
%\section{monolith-demo guide}
%The monolith demo
