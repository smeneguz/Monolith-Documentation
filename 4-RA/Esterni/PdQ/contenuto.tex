%% PIANO DI QUALIFICA

\section{Introduzione}
\subsection{Scopo del documento}


Lo scopo principale di questo documento è di ottenere una buona
qualità, intesa non solo come qualità di prodotto, ma anche come
qualità di processo. Per raggiungere tale obiettivo è necessario un
lavoro intellettuale e arduo poiché la qualità del software è diversa
dagli altri prodotti manifatturieri ed è difficile misurarla in
maniera oggettiva. Per rilevare e successivamente correggere le
anomalie in modo efficace è necessaria un'attività costante di
verifica e validazione da parte del \glossario{team} Obelix.


\subsection{Scopo del prodotto}

Lo scopo del prodotto è quello di creare un \glossario{SDK} che permetta la
realizzazione di cosiddette "bolle interattive" di diversi tipi in
base alle richieste di utenti (che nel nostro caso saranno gli
sviluppatori). Il prodotto è completato con la realizzazione di una
demo presentabile mediante una web app con obiettivo di testare e
provare il corretto funzionamento del SDK.

\subsection{Glossario}
Al fine di evitare ogni ambiguità di linguaggio e massimizzare la
comprensione dei documenti, i termini tecnici, di dominio, gli
acronimi e le parole che necessitano di essere chiarite, sono
riportate nel documento \gloss

\subsection{Riferimenti}

\subsubsection{Normativi}

\begin{itemize}
\item \textbf{Norme di progetto}:  \normediprogetto
\item \textbf{Capitolato d'appalto C5}: \emph{RedBabel, Monolith \url{http://www.math.unipd.it/~tullio/IS-1/2016/Progetto/C5.pdf/}}

\end{itemize}

\subsubsection{Informativi}

\begin{itemize}
\item \textbf{Piano di Progetto}: \pianodiprogetto
\item \textbf{PDCA (Plan-Do-Check-Act): } \emph{\url{http://it.wikipedia.org/wiki/PCDA}}
\item \textbf{Standard ISO/IEC 12207:2008-IEEE Std 12207-2008}: \url{http://ieeexplore.ieee.org/xpl/mostRecentIssue.jsp?punumber=4475822}
\item \textbf{Standard ISO/IEC 15504}:  \url{http://en.wikipedia.org/wiki/ISO/IEC\_15504/}
\item \textbf{Standard ISO/IEC 9126}: \url{http://it.wikipedia.org/wiki/ISO/IEC\_9126}
\item \textbf{Indice di Gulpease}: \url{http://it.wikipedia.org/wiki/Indice\_Gulpease}
\end{itemize}



\section[Visione generale della qualità]{Visione generale della strategia di gestione della qualità}

% Con le strategie utilizzate si cerca di automatizzare il lavoro di
% verifica. Lo scopo è quello di ottenere un riscontro affidabile ed
% adeguato per assicurare un grado di qualità predeterminato e ridurre
% il lavoro manuale permettendo cosi una validazione semplificata.

%%%%%%%%%%%%%%%%%%%%%%%%%%%%%%%%%%%%%%%%%%%%%%%%%%%%%%%%%%%%%%%%%%%
%%%%%%%%%%%%%%%%%%%%%%%%%%%%%%%%%%%%%%%%%%%%%%%%%%%%%%%%%%%%%%%%%%%
%% RIFARE
%%%%%%%%%%%%%%%%%%%%%%%%%%%%%%%%%%%%%%%%%%%%%%%%%%%%%%%%%%%%%%%%%%%
%%%%%%%%%%%%%%%%%%%%%%%%%%%%%%%%%%%%%%%%%%%%%%%%%%%%%%%%%%%%%%%%%%%

\subsection{Obiettivi di qualità}

In questa sezione vengono illustrati gli obiettivi fissati dal gruppo Obelix al fine di garantire la qualità di processo e di prodotto nella realizzazione di Monolith.
Al fine di monitorare costantemente lo stato e il raggiungimento degli obiettivi il gruppo
ha adottato standard e metriche adeguate, nelle sezioni seguenti verranno illustrati in
dettaglio le metodologie, le metriche applicate e le rispettive scale di riferimento.
Sia gli obiettivi che le metriche sono identificati univocamente da un codice alfanumerico
in modo da renderli facilmente tracciabili e quindi controllabili costantemente. La classificazione di obiettivi e metriche è descritta in dettaglio nelle \normediprogetto.

\subsubsection{Qualità di processo}
%% La qualità di processo è definita dallo standard ISO/IEC 15504
%% (SPICE). Questo standard specifica come la qualità è collegata alla
%% maturazione dei processi.
%% Non sarebbe possibile definire la qualità del prodotto senza garantire
%% la qualità del processo. La qualità del prodotto nasce dunque dalla
%% qualità del processo. Per garantire tutto ciò il team Obelix fa uso
%% dello standard ISO/IEC 15504 denominato SPICE, il quale fornisce gli
%% strumenti necessari a valutare l'idoneità dei processi.

La qualità di prodotto non può essere garantita a meno che non sia
garantita la qualità del processo che lo realizza. Nello standard
ISO/IEC 15504 (SPICE) viene definita la qualità di processo e vengono
fornite linee guida per la sua stima. Per maggiori informazioni sullo
standard si veda l'appendice A.1.
Adottando lo standard il gruppo Obelix sarà in grado di stimare
il livello di maturità raggiunto dal suo lavoro.

Nelle revisioni di progettazione il committente
fornisce una valutazione precisa del lavoro svolto fino a quel
momento.
Grazie a questo meccanismo e a valutazioni interne al gruppo è
possibile migliorare il metodo di lavoro.

\'E fondamentale inoltre rispettare i tempi stabiliti nel \pianodiprogetto. Per garantire una corretta stima è necessario misurare
eventuali discostamenti dalla pianificazione.

\paragraph{Miglioramento costante}

Tra una revisione di progettazione e la successiva è possibile
mettere a frutto quanto appreso nel periodo precedente.
Viene adottato il metodo manageriale PDCA (Plan,Do,Check,Act) che
permette l'uso di un approccio ingegneristico in un'attività generica.
In particolare si prescrive la misurazione di grandezze ritenute
rilevanti nello svolgimento delle attività al fine di averne una stima
oggettiva.
In seguito le informazioni così raccolte possono essere utilizzate per
individuare cambiamenti desiderabili nello svolgimento delle attività.
Per una descrizione più dettagliata del metodo PDCA si faccia
riferimento all'appendice \emph{A.2}


\paragraph{Rispetto della pianificazione}
Per capire se le attività di un processo sono in ritardo rispetto a
quanto pianificato all'interno del documento \emph{\pianodiprogetto} viene utilizzata la metrica Schedule Variance. Si desidera che
il ritardo accumulato sia minore del 5\% rispetto al totale
pianificato. Sarebbe invece ottimale essere esattamente in linea con
quanto prevede il \emph{\pianodiprogetto}, o essere addirittura
in anticipo.

\paragraph{Rispetto del budget}
Viene utilizzata la metrica Budget Variance Per capire se i costi di
un processo rientrano nel budget previsto dal \emph{\pianodiprogetto}.
L'obiettivo minimo è quello di avere dei costi che non superano il
budget a disposizione per più del 10\%. Sarebbe invece ottimale che i
costi fossero esattamente in linea con il preventivo o addirittura
minori.

\subsubsection{Qualità di prodotto}
Lo svolgimento del progetto prevede la realizzazione di due prodotti:
i documenti e il software Monolith.
I documenti devono essere precisi e corretti nella forma e nei
contenuti in modo che possano essere un supporto efficace allo
svolgimento del lavoro.

Per garantire la qualità del software il team Obelix cercherà
di aderire al meglio allo standard di qualità ISO/IEC 9126 che
evidenzia quelli che possono essere gli aspetti rilevanti nella
valutazione della qualità di un software.

Per garantire la qualità del prodotto esistono i seguenti processi:
\begin{itemize}
\item \textbf{SQA: Software Quality Assurance} l'insieme delle
  attività realizzate al fine di garantire il raggiungimento degli
  obiettivi di qualità. \'E importante che tale processo sia
  preventivo e non correttivo
\item \textbf{Verifica} assicura che l'esecuzione delle attività dei
  processi svolti non introduca errori
  nel prodotto. Durante l'intera durata del progetto verranno svolte
  attività di verifica sugli
  output dei processi, accertando che esso sia corretto, completo e
  rispetti regole, convenzioni
  e procedure
\item \textbf{Validazione} la conferma oggettiva che assicura che i prodotti finali soddisfino i requisiti
  e le aspettative attese
\end{itemize}

\paragraph{Qualità dei documenti}
Gli obiettivi della qualità dei documenti che il team Obelix desidera raggiungere sono i seguenti:
\begin{itemize}
\item i documenti devono essere il più possibile comprensibili
\item i documenti devono essere corretti a livello ortografico
\item i documenti non devono contenere concetti errati
\end{itemize}

\subparagraph{Leggibilità e comprensibilità}
Per stimare la leggibilità e la comprensibilità dei documenti
viene utilizzato l'indice Gulpease. \'E desiderabile che i documenti
abbiano costantemente un indice maggiore
di 40 come soglia accettabile e 60 come ottimale. Per una descrizione
dettagliata della metrica utilizzata si faccia riferimento
alla metrica "Indice di Gulpease" alla sezione
\emph{2.8.2}. %%%%%%%%%%%%%%%%%%%%%%%%%%%%%%%%%%%%%%%%%%%%%%%%%%%%%%%%


\paragraph{Qualità del software}

Al fine di garantire la qualità del prodotto software, il gruppo ha deciso di adottare lo standard
ISO/IEC 9126, esso classifica la qualità del software e definisce delle metriche utili alla sua
misurazione. In particolare il gruppo si prefigge di garantire le seguenti qualità per il prodotto
software:
\begin{itemize}
\item deve possedere le funzionalità descritte dai requisiti obbligatori
\item deve possedere le funzionalità descritte dai requisiti desiderabili
\item il codice deve risultare mantenibile e facilmente comprensibile
\item deve risultare affidabile e robusto
\item deve essere testato in ogni sua parte per garantirne il
  funzionamento
\end{itemize}
\subparagraph{Funzionalità obbligatorie}
Il prodotto deve implementare tutte le funzionalità descritte dai
requisiti obbligatori. Per monitorare lo stato di implementazione di
tali funzionalità si rapportano i requisiti obbligatori
completati con quelli ancora da completare. A tal fine viene
realizzato il tracciamento dei requisiti sui componenti
software\footnote{presente nella \definizionediprodotto}.
\subparagraph{Funzionalità desiderabili}
Il prodotto deve implementare la maggior parte possibile delle
funzionalità descritte dai requisiti desiderabili. Per monitorare lo
stato di implementazione di tali funzionalità si rapportano i
requisiti desiderabili
completati con quelli ancora da completare.
\subparagraph{Caratteristiche misurabili del software}
La qualità del software si misura attraverso l'adozione di metriche
scelte perché ritenute rilevanti per descrivere le proprietà
critiche del software.
Oltre alle proprietà intrinseche del prodotto viene anche misurata la
qualità dei test, ovvero quanto i test effettivamente verifichino
l'assenza di anomalie nel software.

\subsubsection{Tabella degli Obiettivi}

\begin{center}
	\begin{longtable}{|
			*{1}{>{\centering\arraybackslash}p{1.7 cm}|}
			*{1}{>{\centering\arraybackslash}p{2.3 cm}|}
			*{1}{>{\centering\arraybackslash}p{4.0 cm}|}
			*{1}{>{\centering\arraybackslash}p{2.4 cm}|}}
		\hline
		\textbf{ID} & \textbf{Nome} & \textbf{Descrizione} & \textbf{Metriche}
		\\
		\hline \endhead
		\hline \endfoot
		
		\hline OQD001 & Leggibilità dei documenti & I documenti devono essere leggibili e comprensibili da persone con licenza media superiore & MD001: Gulpease  \\
		\hline OQD002 & Correttezza ortografica dei documenti & I documenti non devono presentare errori ortografici o grammaticali & MD002: Errori ortografici corretti  \\
		\hline OQD003 & Produttività nella stesura dei documenti & Il numero di parole scritte in un'ora deve essere >= 100  & MD003: Produttività di documentazione  \\
		\hline OQS001 & Produttività nella stesura del codice & Il numero di linee di codice scritte in un'ora deve essere >= 100  & MS001: Produttività di codifica  \\
		\hline OQS002 & Copertura del codice & Monolith deve essere testato in ogni su parte per garantire le funzionalità previste dai requisiti  & MS002: Statement Coverage  \\
		\hline  &  &  & MS003: Branch Coverage  \\
		\hline OQS003 & Copertura dei test & La percentuale di superamento dei test deve essere >= 80\% del totale  & MS004: Percentuale superamento test  \\
		\hline OQS004 & Manutenzione e comprensibilità del codice & Il codice di Monolith deve essere quanto più comprensibile e manutenibile  & MS005: Complessità ciclomatica media  \\
		\hline  &  &  & MS006: Numero di metodi per package (max)  \\
		\hline  &  &  & MS007: Variabili non utilizzate e non definite  \\
		\hline  &  &  & MS008: Numero di parametri per metodo (max)  \\
		\hline  &  &  & MS009: Halstead difficulty per-function (media)  \\
		\hline  &  &  & MS010: Halstead volume per-function (media)  \\
		\hline  &  &  & MS011: Halstead effort per-function   \\
		\hline  &  &  & MS012: Maintainability index  \\
		\hline
		
	\end{longtable}
\end{center}





%% \subsection{Procedure di controllo della qualità di processo}

%% Le linee guida per la gestione della qualità di processo seguono il
%% modello \glossario{PDCA} descrivendo come devono essere attuate le
%% procedure di controllo:

%% \begin{itemize}
%% \item Pianificazione dettagliata
%% \item Monitoraggio delle attività pianificate
%% \item Definizione delle risorse necessarie al conseguimento degli obiettivi
%% \item Utilizzo di metriche per verificare il miglioramento delle qualità dei processi
%% \end{itemize}

%% \subsection{Procedure di controllo della qualità di prodotto}

%% Con i seguenti processi verrà garantita il controllo della qualità del prodotto:
%% \begin{itemize}
%% \item \textbf{Software Quality Assurance (SQA)}: è l'insieme delle attività che serve per garantire il raggiungimento degli obiettivi di qualità
%% \item \textbf{Verifica}: assicura che non siano stati introdotti errori nel prodotto con l'esecuzione dei processi
%% \item  \textbf{Validazione}: è la conferma oggettiva che assicura che i prodotti finali soddisfino i requisiti e le aspettative attese
%% \end{itemize}

\subsection{Organizzazione temporale}

%Viene verificato la qualità dei singoli processi e dei loro output.

\subsubsection{Analisi}

In questo periodo verrà verificata la corrispondenza tra i casi d'uso
e i requisiti. Verrà inoltre controllato il rispetto dei processi e
della documentazione prodotta.

\subsubsection{Progettazione Architetturale}

In questo periodo avviene la verifica dei processi relativi all'analisi e
ai nuovi documenti di progettazione. Inoltre si verifica che i test
siano adeguatamente pianificati come descritti ed eseguiti nel documento \normediprogetto .

\subsubsection{Progettazione di Dettaglio e Codifica}

In questo periodo avviene la verifica dei processi relativi alla
progettazione insieme alla verifica delle attività di codifica tramite
tecniche di analisi statica e dinamica.



\subsection{Tecniche di analisi}

\subsubsection{Analisi statica}
L'analisi statica è una tecnica di analisi che si applica sia alla
documentazione che al codice e permette di individuare errori ed
anomalie. Essa si può svolgere in due modi distinti che sono Walkthrough ed
Inspection.

\paragraph{Walkthrough}
\'E una tecnica che viene utilizzata soprattutto nelle prime attività
del progetto quando ancora non è presente una adeguata esperienza dei
membri del gruppo che permetta di attuare una verifica più
mirata e precisa.
Con l'utilizzo di questa tecnica, il \emph{Verificatore} sarà in grado
di stilare una lista di controllo con gli errori più frequenti in modo
da favorire il miglioramento di tale attività nel lavoro futuro.
Walkthrough è un'attività onerosa e collaborativa che richiede
l'intervento di più persone per essere efficace ed efficiente. Dopo
una prima fase di lettura e individuazione degli errori, segue una
fase di discussione con la finalità di esaminare i difetti riscontrati
e di proporre le dovute correzioni. L'ultima fase consiste nel
correggere gli errori rilevati e nello scrivere un rapporto che
elenchi le modifiche effettuate.


\paragraph{Inspection}
L’inspection consiste nell'analisi mirata di alcune parti dei
documenti o del codice ritenute maggior fonte di errore. Deve essere
seguita una lista di controllo per svolgere efficacemente questa
attività; tale lista deve essere redatta anticipatamente ed è
sostanzialmente frutto dell'esperienza maturata dai membri del team
con tecniche di walkthrough. L’inspection è dunque più rapida del
walkthrough, in quanto il prodotto viene analizzato solo in alcune
sue parti e con una lista di controllo ben precisa.

\subsubsection{Analisi dinamica}

L'analisi dinamica si applica solamente al prodotto software e viene
svolta durante l'esecuzione del codice mediante l'uso di test
progettati per rilevare la presenza di difetti.
L'obiettivo del test del software è infatti, quello di
realizzare un prodotto il più possibile esente da errori. Il
principale ostacolo alla fase di test è sintetizzato nella tesi di
Dijkstra, la quale afferma che il test può indicare la presenza di
errori, ma non ne può garantire l'assenza.
Affinché tale attività sia utile e generi risultati attendibili è
necessario che i test effettuati siano ripetibili: dato un certo input
deve essere prodotto sempre uno stesso output in uno specifico
ambiente. Di conseguenza, i tre elementi fondamentali di un test sono:

\begin{itemize}
\item \textbf{Ambiente}: sistema hardware e software sui quali è stato
  pianificato l'utilizzo del prodotto software sviluppato. Su di essi
  deve essere specificato uno stato iniziale dal quale poter eseguire
  il test

\item \textbf{Specifica}: definizione di input e output
\item \textbf{Procedure}: definizione di come devono essere svolti i
  test, in che ordine devono essere eseguiti e come devono essere
  analizzati i risultati
\end{itemize}




%%%%%%%%%%%%%%%%%%%%%%%%%%%%%%%%%%%%%%%%%%%%%%%%%%%%%%%%%%%%%%%%%%%%%%%%%%%%%%%%%%%%%%%%%%%%%%%%%%%%
\section[Visione di dettaglio della qualità]{Strategia di gestione
  della qualità nel dettaglio}

Il \pianodiprogetto  fissa una serie di scadenze improrogabili e quindi risulta necessario definire
con chiarezza una strategia di qualifica efficace. Gli incrementi della documentazione o del codice
possono essere di natura programmata, quindi prefissati nel calendario, oppure possono insorgere
inaspettatamente. In questo caso sarà necessario programmare le dovute
modifiche.
La qualità di ogni incremento è basata sul rispetto delle \normediprogetto in quanto esse derivano da considerazioni di qualità.
Parte significativa del lavoro
verrà svolto con l'aiuto di automatismi che segnaleranno le problematiche rilevate in modo da
permettere una rapida correzione. L’utilizzo di software apposito permette di eseguire controlli
mirati minimizzando l'utilizzo di risorse umane. L'implementazione di tali controlli viene descritta nelle
\normediprogetto .



\subsection{Responsabilità}

La responsabilità delle verifiche è attribuita al  \emph{Responsabile di
  progetto} e ai  \emph{Verificatori} . All'interno del \pianodiprogetto  sono definiti i compiti e le modalità di
attuazione.

\subsection{Risorse}
Il funzionamento del processo di verifica è garantito grazie al consumo di risorse, distinguibili
nelle categorie a seguire.
\subsubsection{Risorse umane}

Hanno maggiore responsabilità per l'attività di verifica
e validazione il \emph{Responsabile del progetto} e il
\emph{Verificatore}. Per una dettagliata descrizione dei ruoli e delle
loro responsabilità bisogna fare riferimento alle \normediprogetto. Per una dettagliata descrizione dell'impiego delle
risorse umane nell'arco del progetto bisogna riferimento al \pianodiprogetto.

\subsubsection{Risorse tecnologiche}
Per risorse tecnologiche si intendono
tutti gli strumenti software e hardware che il gruppo Obelix intende
utilizzare per attuare le attività di verifica su processi e
prodotti. Per una dettagliata e accurata descrizione di tali strumenti
si faccia riferimento alle \normediprogetto.



\subsection{Misure e metriche}
Con lo scopo di poter monitorare in modo consapevole l'andamento dei
processi e la qualità del prodotto si utilizzano delle metriche per
rendere misurabili e valutabili in modo oggettivo alcune
caratteristiche di documenti, processi e software.


\subsubsection{Metriche per i processi}
Le metriche per i processi hanno lo scopo  di monitorare e rendere
prevedibile l'andamento delle variabili di maggior criticità del
progetto: tempo e costo. Le metriche sono utilizzate in modo consultivo
per consentire un riscontro immediato sullo stato attuale del
progetto; ad ogni incremento verranno valutati tali indici e, se
necessario, verranno stabiliti opportuni provvedimenti da parte del  \emph{Responsabile di progetto}.

\paragraph{Schedule Variance}
Permette di calcolare le tempistiche rispetto l'organizzazione delle attività pianificate alla data
corrente. \'E un indicatore di efficacia soprattutto a beneficio del
cliente.
$$
SV = BCWP − BCWS
$$
Dove:
\begin{itemize}
\item \textbf{BCWP}: indica il valore delle attività realizzate alla data corrente
\item \textbf{BCWS}: indica il costo pianificato per realizzare le attività di progetto alla data corrente
\end{itemize}
Quindi con:
\begin{itemize}
\item $SV>0$: il lavoro prodotto è in anticipo rispetto quanto pianificato
\item $SV<0$: il lavoro è in ritardo
\item $SV=0$: il lavoro è in linea con quanto stabilito
\end{itemize}

\paragraph{Budget Variance}
Permette di calcolare i costi rispetto alla data corrente. \'E un indicatore che ha un valore
unicamente contabile e finanziario.
$$
BV = BCWS − ACWP
$$
Dove:
\begin{itemize}
\item \textbf{BCWS}: indica il costo pianificato per realizzare le attività di progetto alla data corrente
\item \textbf{ACWP}: indica il costo effettivamente sostenuto per realizzare le attività di progetto alla
  data corrente
\end{itemize}
Quindi:
\begin{itemize}
\item $BV>0$: il budget speso è minore di quanto pianificato
\item $BV<0$: il budget speso è maggiore di quanto pianificato
\item $BV=0$: il budget speso è in linea con quanto stabilito
\end{itemize}

\paragraph{Produttività}

\subparagraph{Produttività di documentazione}
Indica la produttività media nel redigere i documenti.
$$
\text{Produttività di documentazione} = \frac{\text{Parole}}{\text{Ore persona}}
$$
Dove:
\begin{itemize}
\item \textbf{Parole}: indica il numero di parole presente nei documenti
\item \textbf{Ore persona}: indica il numero di ore produttive impiegate per
  realizzare tali documenti
\end{itemize}

\subparagraph{Parametri utilizzati}
\begin{itemize}
\item Range-ottimale:$ [≥100]$
\end{itemize}



\subparagraph{Produttività di codifica}
Indica la produttività media nelle attività di codifica.
$$
\text{Produttività di codifica} = \frac{\text{LOCs}}{\text{Ore persona}}
$$
Dove:
\begin{itemize}
\item \textbf{LOCs}: indica il numero di linee di codice prodotte
  (Lines Of Code)
\item \textbf{Ore persona}: indica il numero di ore produttive
  impiegate per produrre tale codice
\end{itemize}
\subparagraph{Parametri utilizzati}
\begin{itemize}
	\item Range-ottimale:$ [≥100]$
\end{itemize}


\subsubsection{Metriche per i prodotti}

\paragraph{Metriche per i documenti}
I documenti sono di qualità se sono leggibili. La leggibilità è
misurabile tramite un indice calcolabile sulle caratteristiche del
testo.

\subparagraph{Indice Gulpease}
L'indice Gulpease misura la leggibilità di un testo in Italiano
utilizzando variabili linguistiche come la lunghezza delle parole e
delle frasi.

$$
G = 89 + \frac{300 \times \text{numero delle frasi} - 10 \times \text{numero delle lettere}}{\text{numero
    delle parole}}
$$

Il valore risultante è un numero compreso tra 0 e 100, dove 100 indica
la leggibilità massima e 0 la leggibilità minima. I documenti sono da
considerare secondo le seguenti fasce:
\begin{itemize}
\item $ G < 80 $ sono considerati difficili per chi ha la sola licenza
  elementare
\item $ G < 60 $ sono considerati difficili per chi ha la sola licenza
  media
\item $ G < 40 $ sono considerati difficili per chi ha un diploma di
  scuola  superiore
\end{itemize}

\textbf{Parametri utilizzati}\\
\begin{itemize}
\item Range di accettazione [40-90]
\item Range ottimale [50-90]
\end{itemize}







\paragraph{Metriche per software}
%%%%%%%%%%%%%%%%%%%%%%%%%%%%%%%% INTRO
Di seguito vengono elencate le metriche per il software prodotto e le relative caratteristiche di
qualità che intendono valutare.

\begin{center}

  \begin{tabular}{|l|c|}
   % {|>{\centering}m{6cm} ||>{\centering}m{6cm}|}
    \hline
    \textbf{Metriche scelte} & \textbf{Caratteristiche di qualità} \\
    \hline
    %Copertura requisiti obbligatori  & Funzionalità \\
    %Copertura requisiti desiderabili & Funzionalità \\
    Complessità ciclomatica  & Manutenibilità \\
    Numero di metodi per package & Manutenibilità \\
    Variabili non utilizzate e non definite & Manutenibilità \\
    Numero di parametri per metodo & Manutenibilità \\
    Metriche di Halstead  & Manutenibilità \\
    Maintainability index  & Manutenibilità \\
    Statement coverage & Affidabilità \\
    Branch coverage & Affidabilità \\
    Percentuale superamento test & Affidabilità \\
    \hline
  \end{tabular}
  \captionof{table}{Scopo delle metriche adottate}
\end{center}


\subparagraph{Complessità ciclomatica}
La complessità ciclomatica è una metrica software che indica la complessità di un programma
misurando il numero di cammini linearmente indipendenti attraverso il grafo di controllo di flusso.
Nel grafo sopracitato i nodi corrispondono a gruppi indivisibili di istruzioni, mentre gli archi
connettono due nodi se il secondo gruppo di istruzioni può essere eseguito immediatamente dopo
il primo gruppo. Tale indice può essere applicato indistintamente a singole funzioni, moduli,
metodi o \glossario{package} di un programma. Si vuole utilizzare tale metrica per limitare la complessità
durante le attività di sviluppo del prodotto software. Può rivelarsi utile durante il testing per
determinare il numero  di test necessari, essendo l'indice di complessità un limite superiore
al numero di test necessari per raggiungere la copertura completa. \\

Parametri utilizzati:
\begin{itemize}
\item Range-accettazione: $[0 - 25]$
\item Range-ottimale: $[0 - 10]$
\end{itemize}

\subparagraph{Numero di metodi - NOM}
Il \emph{Number of methods} è una metrica usata per calcolare la media
delle occorrenze dei metodi per package. Un package non dovrebbe
contenere un numero eccessivo di metodi. Valori superiori al range
ottimale massimo potrebbero indicare una necessità di maggiore
scomposizione del package. \\

Parametri utilizzati:
\begin{itemize}
\item Range-accettazione: $[3 - 15]$
\item Range-ottimale: $[3 - 7]$
\end{itemize}

\subparagraph{Variabili non utilizzate e non definite}
La presenza di variabili non utilizzate viene considerata pollution, pertanto non viene tollerata.
Tali occorrenze vengono rilevate analizzando l’Abstract syntax tree
eseguendo un confronto tra le variabili  dichiarate e quelle
inizializzate. Per sua natura, \glossario{Javascript} non blocca
l'insorgenza di tali occorrenze, pertanto si rischia di dichiarare una
variabile  e poi utilizzarne una
con nome leggermente diverso, oppure semplicemente dichiarare una variabile che in seguito non
verrà mai utilizzata.\\

{Parametri utilizzati:
  \begin{itemize}
  \item  Range-accettazione: $[0 - 0]$
  \item Range-ottimale: $[0 - 0]$
  \end{itemize}

  \subparagraph{Numero parametri per metodo}
  Un numero elevato di parametri per un metodo potrebbe evidenziare un
  metodo troppo complesso.
  Non c'è una regola forte per il numero di parametri possibili in un
  metodo, ma citando Robert Martin in Clean Code\footnote{Robert Martin, Clean
    Code: A Handbook of Agile Software Craftsmanship. Prentice Hall
    (2008)} :

  \begin{displayquote}
    The ideal number of arguments for a function is zero (niladic). Next
    comes one (monadic), followed  closely by two (dyadic). Three
    arguments (triadic) should be  avoided where possible. More
    than three (polyadic) requires very special justification – and then
    shouldn’t be used anyway.
  \end{displayquote}
  Vengono quindi seguite le linee guida dei seguenti parametri:

  \begin{itemize}
  \item Range-accettazione: $[0 - 8]$
  \item Range-ottimale: $[0 - 4]$
  \end{itemize}

  \subparagraph{Halstead}
  La metrica di \glossario{Halstead} oltre ad essere un indice di complessità, permette di identificare le
  proprietà misurabili del software e le relative relazioni. Si basa
  sulla necessità che una metrica sia indipendente
  dalla specifica piattaforma e dal linguaggio di programmazione.
  Sono identificati i seguenti dati all'interno di un problema astratto:
  \begin{itemize}
  \item $n_1$ : indica il numero di operatori distinti
  \item $n_2$ : indica il numero di operandi distinti
  \item $N_1$ : indica il numero totale di operatori
  \item $N_2$ : indica il numero totale di operandi
  \end{itemize}
  Da cui si ottiene:
  \begin{itemize}
  \item $n = n_1 + n_2$ : vocabolario della funzione
  \item $N = N_1 + N_2$ : lunghezza della funzione
  \end{itemize}

  Data la scarsa disponibilità in rete di valori di riferimento per il
  Javascript, i range
  saranno da valutare in un momento successivo alla RR. \\

  \textbf{Halstead difficulty per-function}
  Il livello di difficoltà di una funzione misura la propensione all'errore ed è proporzionale al numero
  di operatori presenti.

  $$
  D = (\frac{n_1}{2})\times(\frac{N_2}{n_2})
  $$

  Parametri utilizzati:
  \begin{itemize}
  \item Range-accettazione: $[0 - 30]$
  \item Range-ottimale: $[0 - 15]$ \\
  \end{itemize}

  \textbf{Halstead volume per-function}
  Il volume descrive la dimensione dell'implementazione di un algoritmo e si basa sul numero di
  operazioni eseguite e sugli operandi di una funzione. Il volume di una function senza parametri
  composta da una sola linea è 20, mentre un indice superiore a 1000 indica che probabilmente la
  funzione esegue troppe operazioni.
  $$
  V = N \times \log_2 n
  $$
  Parametri utilizzati:
  \begin{itemize}
  \item Range-accettazione: $[20 - 1500]$
  \item Range-ottimale: $[20 - 1000]$ \\
  \end{itemize}

  \textbf{Halstead effort per-function}
  Lo sforzo per implementare o comprendere il significato di una funzione è proporzionale al volume
  e al suo livello di difficoltà.
  $$
  E = V \times D
  $$ \\
  Parametri utilizzati:
  \begin{itemize}
  \item Range-accettazione: $[0 - 400]$
  \item Range-ottimale: $[0 - 300]$
  \end{itemize}

  \subparagraph{Maintainability index}
  Questa metrica\footnote{Definita nel 1991 da Paul Oman e Jack
    Hagemeister alla University of Idaho.}
  è una scala logaritmica da $-\infty$ a 171, calcolata sulla base delle linee di codice
  logiche, della complessità ciclomatica e dall’indice Halstead effort. Un valore alto indica una
  maggiore manutenibilità.\\

  Parametri utilizzati:
  \begin{itemize}
  \item Range-accettazione: $[>70]$
  \item Range-ottimale: $[>90]$\\
  \end{itemize}



  \subparagraph{Statement Coverage}
  Permette di calcolare quante linee di codice di ciascun modulo delle unità sono eseguite almeno
  una volta nell'esecuzione dei test. Tale metrica è espressa in percentuale.\\

  Parametri utilizzati:
  \begin{itemize}
  \item Range-accettazione: $[70 - 100]$
  \item Range-ottimale: $[85 - 100]$
  \end{itemize}

  \subparagraph{Branch Coverage}
  Permette di calcolare quanti rami della logica di flusso sono attraversati almeno una volta durante
  l'esecuzione dei test. Tale metrica è espressa in percentuale.\\

  Parametri utilizzati:
  \begin{itemize}
  \item Range-accettazione: $[70 - 100]$
  \item Range-ottimale: $[85 - 100]$
  \end{itemize}

  \subparagraph{Percentuale superamento test}
Questa metrica esamina la percentuale di successo dei test ricavati dai requisiti e dalle relative
funzionalità che il software dovrà ottenere. Indica infatti la percentuale dei test eseguiti con
successo.

$$
\text{Percentuale superamento test} = 
\frac{\text{Numero test superati} \times 100}
{\text{Numero test pianificati}}
$$

\subparagraph{Parametri utilizzati}
\begin{itemize}
	\item Range-accettazione: $[70 - 100]$
	\item Range-ottimale: $[80 - 100]$
\end{itemize}

\subsection{Riassunto della relazione tra metriche e obiettivi di qualità}

\begin{center}
	\begin{longtable}{|
			*{1}{>{\centering\arraybackslash}p{1.7 cm}|}
			*{1}{>{\centering\arraybackslash}p{2.3 cm}|}
			*{1}{>{\centering\arraybackslash}p{4.0 cm}|}
			*{1}{>{\centering\arraybackslash}p{2.4 cm}|}}
		\hline
		\textbf{ID} & \textbf{Nome} & \textbf{Soglie di accettabilità} & \textbf{Obiettivi}
		\\
		\hline \endhead
		\hline \endfoot
		
		\hline MD001 & Gulpease & \makecell{\textbf{Valore minimo}: \\ >= 40 \\ \textbf{Valore ottimale}: \\ >= 50} &  OQD001: Leggibilità dei documenti  \\
		\hline MD002 & Errori ortografici corretti & \makecell{\textbf{Valore minimo}: \\ 100 \% corretti \\ \textbf{Valore ottimale}: \\ 100 \% corretti} &  OQD002: Correttezza ortografica dei documenti  \\
		\hline MD003 & Produttività di documentazione & \makecell{\textbf{Valore minimo}: \\ 100  \\ \textbf{Valore ottimale}: \\ >=100} &  OQD003: Produttività nella stesura dei documenti  \\
		\hline MS001 & Produttività di codifica & \makecell{\textbf{Valore minimo}: \\ 100  \\ \textbf{Valore ottimale}: \\ >=100} &  OQS001: Produttività nella stesura del codice \\ 
		\hline MS002 & Statement Coverage & \makecell{\textbf{Valore minimo}: \\ >= 70  \\ \textbf{Valore ottimale}: \\ >=85} &  OQS002: Copertura del codice \\ 
		\hline MS003 & Branch Coverage & \makecell{\textbf{Valore minimo}: \\ >= 70  \\ \textbf{Valore ottimale}: \\ >=85} &  OQS002: Copertura del codice \\
		\hline MS004 & Percentuale superamento test & \makecell{\textbf{Valore minimo}: \\ >= 80  \\ \textbf{Valore ottimale}: \\ 100} &  OQS003: Copertura dei test \\
		\hline MS005 & Complessità ciclomatica (media) & \makecell{\textbf{Valore minimo}: \\ <=25  \\ \textbf{Valore ottimale}: \\ <=10} &  OQS004: Manutenzione e comprensibilità del codice \\
		\hline MS006 & Numero di metodi per package (max) & \makecell{\textbf{Valore minimo}: \\ <=15  \\ \textbf{Valore ottimale}: \\  <=7} &  OQS004: Manutenzione e comprensibilità del codice \\
		\hline MS007 & Variabile non utilizzate e non definite & \makecell{\textbf{Valore minimo}: \\ 0  \\ \textbf{Valore ottimale}: \\ 0} &  OQS004: Manutenzione e comprensibilità del codice \\
		\hline MS008 & Numero di parametri per metodo (max) & \makecell{\textbf{Valore minimo}: \\ <=8  \\ \textbf{Valore ottimale}: \\ <=4} &  OQS004: Manutenzione e comprensibilità del codice \\
		\hline MS009 & Halstead difficulty per-function (media) & \makecell{\textbf{Valore minimo}: \\ <=30 \\ \textbf{Valore ottimale}: \\ <=15} &  OQS004: Manutenzione e comprensibilità del codice \\
		\hline MS010 & Halstead volume per-function (media) & \makecell{\textbf{Valore minimo}: \\ <=1500 \\ \textbf{Valore ottimale}: \\ <=1000} &  OQS004: Manutenzione e comprensibilità del codice \\
		\hline MS011 & Halstead effort per-function & \makecell{\textbf{Valore minimo}: \\ <=400 \\ \textbf{Valore ottimale}: \\ <=300} &  OQS004: Manutenzione e comprensibilità del codice \\
		\hline MS012 & Maintainability index & \makecell{\textbf{Valore minimo}: \\ >70 \\ \textbf{Valore ottimale}:\\ >90} &  OQS004: Manutenzione e comprensibilità del codice \\
		\hline
		
	\end{longtable}
\end{center}

  %%%%%%%%%%%%%%%%%%%%%%%%%%%%%%%%%%%%%%%%%%%%%%%%%%%%%%%%%%%%%%%%%%%%%%%%%%%%%%

  \section{Gestione amministrativa della revisione}

  \subsection{Comunicazione delle anomalie}
  Identificare le anomalie permette la correzione dei difetti ricercati
  dal processo di Software Quality  Management e informa il  \emph{Responsabile}
  di  progetto sullo stato del prodotto. Analizzare e
  catalogare le anomalie è utile per discutere, durante revisioni e
  riunioni, su quali modifiche e correzioni applicare e con quale priorità. Di seguito è presente la lista delle definizioni di anomalie
  (IEEE 610.12-90) adottate dal gruppo:
  \begin{itemize}
  \item \textbf{Error}: differenza riscontrata tra il risultato di una computazione e il valore teorico atteso
  \item \textbf{Fault}: un passo, un processo o un dato definito in modo
    erroneo. Corrisponde a quanto viene comunemente definito come bug
  \item \textbf{Failure}: il risultato di un fault
  \item \textbf{Mistake}: azione umana che produce un risultato errato
  \end{itemize}


  \subsection{Procedure di controllo per la qualità di processo}
  Le procedure di controllo per la qualità di processo hanno il fine di migliorare la qualità del
  processo e diminuire i costi e tempi di sviluppo. Esistono due
  approcci principali:
  \begin{itemize}
  \item \textbf{A maturità di processo}: riflette le buone pratiche di management e tecniche di sviluppo.
    L'obiettivo primario è la qualità del prodotto e la prevedibilità dei processi
  \item \textbf{Agile}: sviluppo iterativo senza l’overhead della
    documentazione e di tutti gli aspetti predeterminabili. Ha come
    caratteristica la responsività ai cambiamenti dei requisti cliente e
    uno sviluppo rapido.
  \end{itemize}
  Il team adotterà il primo approccio, essendo più adatto ad un gruppo inesperto. Con una visione
  proattiva si cerca di avere maggior controllo e previsione sulle attività da svolgere. Questa viene
  anche indicata come best practice per gruppi poco esperti.
  Il processo con maggiore influenza sulla qualità del sistema non è quello di sviluppo ma quello
  di progettazione. È qui che le capacità e le esperienze dei singoli danno un contributo decisivo.
  Il miglioramento dei processi è un processo ciclico composto da tre
  sotto-processi:

  \begin{itemize}
  \item \textbf{Misurazione del processo}: misura gli attributi del progetto, punta ad allineare gli
    obiettivi con le misurazioni effettuate. Questo forma una \glossario{baseline} che aiuta a capire se i
    miglioramenti hanno avuto effetto
  \item \textbf{Analisi del processo}: vengono identificate le problematiche ed i colli di bottiglia dei
    processi
  \item \textbf{Modifiche del processo}: i cambiamenti vengono proposti in risposta alle problematiche
    riscontrate
  \end{itemize}
  Il team procederà nel seguente modo:
  \begin{itemize}
  \item Nell'appendice del seguente documento verranno inserite le misurazioni
    rilevate sulle metriche descritte nella sezione \emph{2.8.1}
  \item Le modifiche al processo vengono attuate all’inizio del processo incrementale successivo.
    Queste attività sono programmate nel  \emph{\pianodiprogetto}
  \end{itemize}

\section{Pianificazione dei Test}

Vengono riportati e descritti in questa sezione i test da implementare per garantire che il
software prodotto rispecchi le funzionalità a fronte dei risultati attesi. 
Tutte le attività di testing prodotte devono poter essere ripetibili e
devono essere deterministiche, al fine di poter fornire delle
informazioni utili a intraprendere azioni di correzione nel caso in
cui i risultati ottenuti siano diversi da quelli attesi. 
Ogni test è identificato da un codice univoco la cui sintassi viene descritta nel documento \normediprogetto

\subsection{Test di Unità}

Tale tipologia di test serve per testare il corretto funzionamento
delle singole unità, ossia delle più piccole componenti software
singolarmente verificabili. Solitamente l'unità trova corrispondenza
in un metodo di una classe tra quelli descritti nel documento
\definizionediprodotto. Per ogni test viene specificato il proprio
codice univoco, la descrizione, lo stato di implementazione attuale e
il risultato (Superato/Non Superato). 

\begin{center}
\begin{longtable}{|
*{1}{>{\centering\arraybackslash}p{1.3cm}|}
*{1}{>{\centering\arraybackslash}p{5cm}|}
*{1}{>{\centering\arraybackslash}p{2.5cm}|}
*{1}{>{\centering\arraybackslash}p{2.5cm}|}}
\hline \textbf{Test} & \textbf{Descrizione} & \textbf{Stato} & \textbf{Superato} \\
\hline \endhead

TU1 & Verificare che la classe CheckBoxList sia instanziabile. & \textcolor{ForestGreen}{Implementato} & \textcolor{ForestGreen}{Superato}\\
 \hline 
TU2 & Verificare che la classe CheckButton sia instanziabile. & \textcolor{ForestGreen}{Implementato} & \textcolor{ForestGreen}{Superato}\\
 \hline 
TU3 & Verificare che la classe ComboBox sia instanziabile. & \textcolor{ForestGreen}{Implementato} & \textcolor{ForestGreen}{Superato}\\
 \hline 
TU4 & Verificare che la classe Image sia instanziabile. & \textcolor{ForestGreen}{Implementato} & \textcolor{ForestGreen}{Superato}\\
 \hline 
TU5 & Verificare che la classe ImageButton sia instanziabile. & \textcolor{ForestGreen}{Implementato} & \textcolor{ForestGreen}{Superato}\\
 \hline 
TU6 & Verificare che la classe LineEdit sia instanziabile. & \textcolor{ForestGreen}{Implementato} & \textcolor{ForestGreen}{Superato}\\
 \hline 
TU7 & Verificare che la classe LineEditComboBox sia instanziabile. & \textcolor{ForestGreen}{Implementato} & \textcolor{ForestGreen}{Superato}\\
 \hline 
TU8 & Verificare che la classe LineEditPushButton sia instanziabile. & \textcolor{ForestGreen}{Implementato} & \textcolor{ForestGreen}{Superato}\\
 \hline 
TU9 & Verificare che la classe PushButton sia instanziabile. & \textcolor{ForestGreen}{Implementato} & \textcolor{ForestGreen}{Superato}\\
 \hline 
TU10 & Verificare che la classe RadioButtonGroup sia instanziabile. & \textcolor{ForestGreen}{Implementato} & \textcolor{ForestGreen}{Superato}\\
 \hline 
TU11 & Verificare che la classe TextAreaButton sia instanziabile. & \textcolor{ForestGreen}{Implementato} & \textcolor{ForestGreen}{Superato}\\
 \hline 
TU12 & Verificare che la classe TextAreaComboBox sia instanziabile. & \textcolor{ForestGreen}{Implementato} & \textcolor{ForestGreen}{Superato}\\
 \hline 
TU13 & Verificare che un componente LineEdit passi al componente genitore che lo contiene l'input testuale nel modo corretto. & \textcolor{ForestGreen}{Implementato} & \textcolor{ForestGreen}{Superato}\\
 \hline 
TU14 & Verificare che in un componente PushButton venga impostato il nome nel modo corretto. & \textcolor{ForestGreen}{Implementato} & \textcolor{ForestGreen}{Superato}\\
 \hline 
TU15 & Verificare che al click di un componente PushButton vengano eseguiti gli eventi associati alla pressione. & \textcolor{ForestGreen}{Implementato} & \textcolor{ForestGreen}{Superato}\\
 \hline 
TU16 & Verificare che in un componente ComboBox vengano impostate le opzioni (mandate dal componente genitore) nel modo corretto. & \textcolor{ForestGreen}{Implementato} & \textcolor{ForestGreen}{Superato}\\
 \hline 
TU17 & Verificare che un componente ComboBox passi al componente genitore che lo contiene l'opzione selezionata nel modo corretto. & \textcolor{ForestGreen}{Implementato} & \textcolor{ForestGreen}{Superato}\\
 \hline 
TU18 & Verificare che in un componente CheckButton venga cambiato in modo corretto lo stato da "cliccato" a "non cliccato" e viceversa. & \textcolor{ForestGreen}{Implementato} & \textcolor{ForestGreen}{Superato}\\
 \hline 
TU19 & Verificare che ad un componente CheckButton venga associata la label scelta nel modo corretto. & \textcolor{ForestGreen}{Implementato} & \textcolor{ForestGreen}{Superato}\\
 \hline 
TU20 & Verificare che in un componente Image l'immagine venga aggiunta in modo corretto. & \textcolor{ForestGreen}{Implementato} & \textcolor{ForestGreen}{Superato}\\
 \hline 
TU21 & Verificare che in un componente Image la didascalia venga aggiunta in modo corretto. & \textcolor{ForestGreen}{Implementato} & \textcolor{ForestGreen}{Superato}\\
 \hline 
TU22 & Verificare che in un componente Image le dimensioni (larghezza e altezza) vengano impostate in modo corretto. & \textcolor{ForestGreen}{Implementato} & \textcolor{ForestGreen}{Superato}\\
 \hline 
TU23 & Verificare che in un componente ImageButton l'immagine venga aggiunta in modo corretto. & \textcolor{ForestGreen}{Implementato} & \textcolor{ForestGreen}{Superato}\\
 \hline 
TU24 & Verificare che in un componente ImageButton la didascalia venga aggiunta in modo corretto. & \textcolor{ForestGreen}{Implementato} & \textcolor{ForestGreen}{Superato}\\
 \hline 
TU25 & Verificare che in un componente ImageButton le dimensioni (larghezza e altezza) vengano impostate in modo corretto. & \textcolor{ForestGreen}{Implementato} & \textcolor{ForestGreen}{Superato}\\
 \hline 
TU26 & Verificare che al click di un componente ImageButton vengano eseguiti gli eventi associati alla pressione. & \textcolor{ForestGreen}{Implementato} & \textcolor{ForestGreen}{Superato}\\
 \hline 
TU27 & Verificare che al click del PushButton associato al componente LineEditPushButton venga passato al componente genitore l'input testuale del LineEdit nel modo corretto. & \textcolor{ForestGreen}{Implementato} & \textcolor{ForestGreen}{Superato}\\
 \hline 
TU28 & Verificare che in un componente TextAreaButton le dimensioni (larghezza e altezza) vengano impostate nel modo corretto. & \textcolor{ForestGreen}{Implementato} & \textcolor{ForestGreen}{Superato}\\
 \hline 
TU29 & Verificare che al click del PushButton associato al componente TextAreaButton venga passato al componente genitore l'input testuale della textarea nel modo corretto. & \textcolor{ForestGreen}{Implementato} & \textcolor{ForestGreen}{Superato}\\
 \hline 
TU30 & Verificare che in un componente CheckBoxList il numero di opzioni presenti sia corretto. & \textcolor{ForestGreen}{Implementato} & \textcolor{ForestGreen}{Superato}\\
 \hline 
TU31 & Verificare che in un componente RadioButtonGroup il numero di opzioni presenti sia corretto. & \textcolor{ForestGreen}{Implementato} & \textcolor{ForestGreen}{Superato}\\
 \hline 
TU32 & Verificare che un componente RadioButtonGroup passi al componente genitore che lo contiene l'opzione selezionata nel modo corretto. & \textcolor{ForestGreen}{Implementato} & \textcolor{ForestGreen}{Superato}\\
 \hline 
TU33 & Verificare che la classe Check sia instanziabile. & \textcolor{ForestGreen}{Implementato} & \textcolor{ForestGreen}{Superato}\\
 \hline 
TU34 & Verificare che la classe BubbleDatabase sia instanziabile. & \textcolor{ForestGreen}{Implementato} & \textcolor{ForestGreen}{Superato}\\
 \hline 
TU35 & Verificare che la classe BubbleMenu sia instanziabile. & \textcolor{ForestGreen}{Implementato} & \textcolor{ForestGreen}{Superato}\\
 \hline 
TU36 & Verificare che la classe ConfigArea sia instanziabile. & \textcolor{ForestGreen}{Implementato} & \textcolor{ForestGreen}{Superato}\\
 \hline 
TU37 & Verificare che la classe SentBubbles sia instanziabile. & \textcolor{ForestGreen}{Implementato} & \textcolor{ForestGreen}{Superato}\\
 \hline 
TU38 & Verificare che la classe Sidearea1 sia instanziabile. & \textcolor{ForestGreen}{Implementato} & \textcolor{ForestGreen}{Superato}\\
 \hline 
TU39 & Verificare che la classe ReceivedBubble sia instanziabile. & \textcolor{ForestGreen}{Implementato} & \textcolor{ForestGreen}{Superato}\\
 \hline 
TU40 & Verificare che la classe Sidearea2 sia instanziabile. & \textcolor{ForestGreen}{Implementato} & \textcolor{ForestGreen}{Superato}\\
 \hline 
TU41 & Verificare che la classe ContainedElement sia instanziabile. & \textcolor{ForestGreen}{Implementato} & \textcolor{ForestGreen}{Superato}\\
 \hline 
TU42 & Verificare che la classe HorizontalLayout sia instanziabile. & \textcolor{ForestGreen}{Implementato} & \textcolor{ForestGreen}{Superato}\\
 \hline 
TU43 & Verificare che la classe VerticalLayout sia instanziabile. & \textcolor{ForestGreen}{Implementato} & \textcolor{ForestGreen}{Superato}\\
 \hline 
TU44 & Verificare che la classe AbsBubble NON sia instanziabile. & \textcolor{ForestGreen}{Implementato} & \textcolor{ForestGreen}{Superato}\\
 \hline 
TU45 & Verificare che la classe AbsBubbleConfig NON sia instanziabile. & \textcolor{ForestGreen}{Implementato} & \textcolor{ForestGreen}{Superato}\\
 \hline 
TU46 & Verificare che la classe AbsButton NON sia instanziabile. & \textcolor{ForestGreen}{Implementato} & \textcolor{ForestGreen}{Superato}\\
 \hline 
TU47 & Verificare che la classe BubbleCreator NON sia instanziabile. & \textcolor{ForestGreen}{Implementato} & \textcolor{ForestGreen}{Superato}\\
 \hline 
TU48 & Verificare che il metodo validate della classe Check effettui la validazione dell'oggetto in modo corretto. & \textcolor{ForestGreen}{Implementato} & \textcolor{ForestGreen}{Superato}\\
 \hline 
TU49 & Verificare che il metodo insertBubble effettui l'inserimento dei dati nel database correttamente. & \textcolor{ForestGreen}{Implementato} & \textcolor{ForestGreen}{Superato}\\
 \hline 
TU50 & Verificare che il metodo updateBubble effettui la modifica dei dati nel database correttamente. & \textcolor{ForestGreen}{Implementato} & \textcolor{ForestGreen}{Superato}\\
 \hline 
TU51 & Verificare che il metodo removeBubble effettui la rimozione dei dati dal database correttamente. & \textcolor{ForestGreen}{Implementato} & \textcolor{ForestGreen}{Superato}\\
 \hline 
\end{longtable}
\captionof{table}{Test di unità}
\end{center}

\subsection{Test di Integrazione}

Tale tipologia di test serve per verificare che le varie componenti
del sistema software interagiscano tra loro nel modo atteso. Per ogni
test viene specificato il proprio codice univoco, la descrizione, il
componente cui si fa riferimento, lo stato di implementazione attuale
e il risultato (Superato/Non Superato). 

\begin{center}
\begin{longtable}{|
*{1}{>{\centering\arraybackslash}p{1cm}|}
*{1}{>{\centering\arraybackslash}p{7cm}|}
*{1}{>{\centering\arraybackslash}p{3cm}|}}
\hline \textbf{Test} & \textbf{Descrizione} & \textbf{Stato}\\
\hline \endhead

TI1 & \textbf{Monolith::UI::Layouts}: Verificare che VerticalLayout disponga i componenti uno sotto l'altro. & \textcolor{ForestGreen}{Implementato}\\
 \hline 
TI2 & \textbf{Monolith::UI::Layouts}: Verificare che HorizontalLayout disponga i componenti uno accanto all'altro. & \textcolor{ForestGreen}{Implementato}\\
 \hline 
TI3 & \textbf{Monolith::UI::uiConstruction}: Verificare che BubbleDiscriminator assegni correttamente al tipo di BubbleCreator il nome scelto. & \textcolor{ForestGreen}{Implementato}\\
 \hline 
TI4 & \textbf{Monolith::UI::uiConstruction}: Verificare che BubbleDiscriminator selezioni il sottotipo di BubbleCreator corretto per la creazione del menu di configurazione del tipo di bolla selezionato. & \textcolor{ForestGreen}{Implementato}\\
 \hline 
TI5 & \textbf{Monolith::UI::uiConstruction}: Verificare che BubbleDiscriminator selezioni il sottotipo di BubbleCreator corretto per la creazione di un'istanza del tipo di bolla inviata. & \textcolor{ForestGreen}{Implementato}\\
 \hline 
TI6 & \textbf{Monolith::UI::uiConstruction}: Verificare che BubbleDiscriminator selezioni il sottotipo di BubbleCreator corretto per la creazione di un'istanza del tipo di bolla ricevuta. & \textcolor{ForestGreen}{Implementato}\\
 \hline 
TI7 & \textbf{Monolith::UI::uiConstruction}: Verificare che BubbleDiscriminator selezioni il sottotipo di BubbleCreator corretto per la creazione del pulsante relativo al tipo di bolla selezionato. & \textcolor{ForestGreen}{Implementato}\\
 \hline 
TI8 & \textbf{Monolith::SideAreas:: SideArea1\_pkg}: Verificare che SentBubbles invochi correttamente useDoMakeBubbleSender di BubbleDiscriminator per la creazione dello storico delle bolle inviate. & \textcolor{ForestGreen}{Implementato}\\
 \hline 
TI9 & \textbf{Monolith::SideAreas:: SideArea1\_pkg}: Verificare che BubbleMenu invochi correttamente useDoMakeButton di BubbleDiscriminator per la creazione del menu contenente i pulsanti relativi ai diversi tipi di bolle. & \textcolor{ForestGreen}{Implementato}\\
 \hline 
TI10 & \textbf{Monolith::SideAreas:: SideArea1\_pkg}: Verificare che SideArea1 contenga BubbleMenu, ConfigArea e SentBubbles. & \textcolor{ForestGreen}{Implementato}\\
 \hline 
TI11 & \textbf{Monolith::SideAreas:: SideArea2\_pkg}: Verificare che ReceivedBubble invochi correttamente useDoMakeBubbleReceiver di BubbleDiscriminator per la creazione dello storico delle bolle ricevute. & \textcolor{ForestGreen}{Implementato}\\
 \hline 
TI12 & \textbf{Monolith::SideAreas:: SideArea2\_pkg}: Verificare che SideArea2 contenga ReceivedBubble. & \textcolor{ForestGreen}{Implementato}\\
 \hline 
TI13 & \textbf{CurrencyBubble}: Verificare che CurrencyCreator crei la corretta istanza della bolla di tipo convertitore inviata. & \textcolor{ForestGreen}{Implementato}\\
 \hline 
TI14 & \textbf{CurrencyBubble}: Verificare che CurrencyCreator crei la corretta istanza della bolla di tipo convertitore ricevuta. & \textcolor{ForestGreen}{Implementato}\\
 \hline 
TI15 & \textbf{CurrencyBubble}: Verificare che CurrencyCreator crei correttamente il menu di configurazione per la bolla di tipo convertitore. & \textcolor{ForestGreen}{Implementato}\\
 \hline 
TI16 & \textbf{CurrencyBubble}: Verificare che CurrencyCreator crei correttamente il pulsante relativo al tipo di bolla convertitore. & \textcolor{ForestGreen}{Implementato}\\
 \hline 
TI17 & \textbf{ListBubble}: Verificare che ListCreator crei la corretta istanza della bolla di tipo lista inviata. & \textcolor{ForestGreen}{Implementato}\\
 \hline 
TI18 & \textbf{ListBubble}: Verificare che ListCreator crei la corretta istanza della bolla di tipo lista ricevuta. & \textcolor{ForestGreen}{Implementato}\\
 \hline 
TI19 & \textbf{ListBubble}: Verificare che ListCreator crei correttamente il menu di configurazione per la bolla di tipo lista. & \textcolor{ForestGreen}{Implementato}\\
 \hline 
TI20 & \textbf{ListBubble}: Verificare che ListCreator crei correttamente il pulsante relativo al tipo di bolla lista. & \textcolor{ForestGreen}{Implementato}\\
 \hline 
TI21 & \textbf{PollBubble}: Verificare che PollCreator crei la corretta istanza della bolla di tipo sondaggio inviata. & \textcolor{ForestGreen}{Implementato}\\
 \hline 
TI22 & \textbf{PollBubble}: Verificare che PollCreator crei la corretta istanza della bolla di tipo sondaggio ricevuta. & \textcolor{ForestGreen}{Implementato}\\
 \hline 
TI23 & \textbf{PollBubble}: Verificare che PollCreator crei correttamente il menu di configurazione per la bolla di tipo sondaggio. & \textcolor{ForestGreen}{Implementato}\\
 \hline 
TI24 & \textbf{PollBubble}: Verificare che PollCreator crei correttamente il pulsante relativo al tipo di bolla sondaggio. & \textcolor{ForestGreen}{Implementato}\\
 \hline 
TI25 & \textbf{RandomBubble}: Verificare che RandCreator crei la corretta istanza della bolla di tipo dado inviata. & \textcolor{ForestGreen}{Implementato}\\
 \hline 
TI26 & \textbf{RandomBubble}: Verificare che RandCreator crei la corretta istanza della bolla di tipo dado ricevuta. & \textcolor{ForestGreen}{Implementato}\\
 \hline 
TI27 & \textbf{RandomBubble}: Verificare che RandCreator crei correttamente il menu di configurazione per la bolla di tipo dado. & \textcolor{ForestGreen}{Implementato}\\
 \hline 
TI28 & \textbf{RandomBubble}: Verificare che RandCreator crei correttamente il pulsante relativo al tipo di bolla dado. & \textcolor{ForestGreen}{Implementato}\\
 \hline 
TI29 & \textbf{Monolith::checks}: Verificare che registerCheck assegni correttamente il nome al controllo (Check) selezionato. & \textcolor{ForestGreen}{Implementato}\\
 \hline 
TI30 & \textbf{Monolith::checks}: Verificare che execCheck effettui il controllo (Check) sull'oggetto passato in modo corretto. & \textcolor{ForestGreen}{Implementato}\\
 \hline 
TI31 & \textbf{Monolith::checks}: Verificare che execCheck effettui correttamente il controllo (Check) anche in presenza di più tipologie di controllo. & \textcolor{ForestGreen}{Implementato}\\
 \hline 
TI32 & \textbf{Monolith::Database:: databaseInitialization}: Verificare che l'inizializzazione del database avvenga in modo corretto. & \textcolor{ForestGreen}{Implementato}\\
 \hline 
TI33 & \textbf{Monolith::Database}: Verificare che l'inserimento dei dati nel database avvenga in modo corretto. & \textcolor{Red}{Non Implementato}\\
 \hline 
TI34 & \textbf{Monolith::Database}: Verificare che la modifica dei dati nel database avvenga in modo corretto. & \textcolor{Red}{Non Implementato}\\
 \hline 
TI35 & \textbf{Monolith::Database}: Verificare che la rimozione dei dati dal database avvenga in modo corretto. & \textcolor{Red}{Non Implementato}\\
 \hline 
\end{longtable}
\captionof{table}{Test di integrazione}
\end{center}

\subsection{Test di Sistema}

Tale tipologia di test serve per verificare che il comportamento
dinamico complessivo dell'intero sistema sia conforme ai requisiti
definiti nel documento \analisideirequisiti. Per ogni test viene
specificato il proprio codice univoco, la descrizione,lo stato di
implementazione attuale e il risultato (Superato/Non Superato).
I test non implementati sono relativi ai requisiti non obbligatori. 

\begin{center}
\begin{longtable}{|
*{1}{>{\centering\arraybackslash}p{2.8cm}|}
*{1}{>{\centering\arraybackslash}p{6cm}|}
*{1}{>{\centering\arraybackslash}p{3cm}|}}
\hline \textbf{Test} & \textbf{Descrizione} & \textbf{Stato}\\
\hline \endhead

TSObFu10 & Verificare che l'utente possa accedere alle bolle dall'interfaccia di Rocket.Chat tramite l'utilizzo di due pulsanti aggiunti alla tabbar laterale. & \textcolor{ForestGreen}{Implementato}\\
 \hline 
TSObFu10.2 & Verificare che l'utente possa utilizzare la SideArea1 per creare, inviare, visualizzare e modificare le bolle inviate. & \textcolor{ForestGreen}{Implementato}\\
 \hline 
TSObFu10.2.5.1 & Verificare che l'utente visualizzi un messaggio appropriato nel caso non ci siano bolle nello storico in uscita. & \textcolor{ForestGreen}{Implementato}\\
 \hline 
TSObFu10.2.7 & Verificare che l'utente visualizzi un messaggio di errore nel caso in cui la configurazione della bolla non fosse andata a buon fine. & \textcolor{Red}{Non Implementato}\\
 \hline 
TSObFu10.3 & Verificare che l'utente possa utilizzare la SideArea2 per visualizzare e interagire con le bolle ricevute. & \textcolor{ForestGreen}{Implementato}\\
 \hline 
TSObFu10.3.1.1 & Verificare che l'utente visualizzi un messaggio appropriato nel caso in cui non ci siano bolle nello storico in ingresso. & \textcolor{ForestGreen}{Implementato}\\
 \hline 
TSObFu11.2.1.1 & Verificare che lo sviluppatore possa aggiungere ad una bolla un componente grafico tra quelli presenti nel SDK. & \textcolor{ForestGreen}{Implementato}\\
 \hline 
TSObFu11.2.1.2 & Verificare che l'utente possa aggiungere ad una bolla un layout tra quelli presenti nel SDK. & \textcolor{ForestGreen}{Implementato}\\
 \hline 
TSObFu11.2.1.3 & Verificare che lo sviluppatore possa modificare le proprietà di un componente grafico. & \textcolor{ForestGreen}{Implementato}\\
 \hline 
TSObFu11.2.1.6 & Verificare che lo sviluppatore possa inserire il menu di configurazione per la bolla. & \textcolor{ForestGreen}{Implementato}\\
 \hline 
TSObFu11.2.3 & Verificare che lo sviluppatore possa gestire la persistenza dei dati della bolla. & \textcolor{ForestGreen}{Implementato}\\
 \hline 
TSObFu11.2.3.4 & Verificare che lo sviluppatore possa specificare i parametri di accettazione dell'input. & \textcolor{Red}{Non Implementato}\\
 \hline 
TSDeFu24 & Verificare che l'utente riceva notifiche per eventi di rilievo. & \textcolor{Red}{Non Implementato}\\
 \hline 
TSObFu01-cv & Verificare che l'utente possa utilizzare la bolla di tipo convertitore di valuta. & \textcolor{ForestGreen}{Implementato}\\
 \hline 
TSObFu01.4-cv & Verificare che il mittente e il ricevente possano visualizzare gli importi convertiti. & \textcolor{ForestGreen}{Implementato}\\
 \hline 
TSObFu01.7-cv & Verificare che l'utente non possa inviare la bolla nel caso in cui non siano state selezionate le valute in ingresso e in uscita o l'importo da convertire. & \textcolor{ForestGreen}{Implementato}\\
 \hline 
TSOpFu02-cv & Verificare che l'utente possa convertire importi da valori di pacchetti azionari. & \textcolor{Red}{Non Implementato}\\
 \hline 
TSObFu01-dd & Verificare che l'utente possa utilizzare la bolla di tipo estrazione di numero casuale. & \textcolor{ForestGreen}{Implementato}\\
 \hline 
TSObFu01.2-dd & Verificare che il mittente e il ricevente visualizzino il numero casuale generato. & \textcolor{ForestGreen}{Implementato}\\
 \hline 
TSOpFu01.2.1-dd & Verificare che il mittente e il ricevente visualizzino il numero casuale sotto forma di immagine. & \textcolor{Red}{Non Implementato}\\
 \hline 
TSObFu01.6-dd & Verificare che l'utente non possa inviare la bolla senza aver impostato il range. & \textcolor{ForestGreen}{Implementato}\\
 \hline 
TSObFu01-ls & Verificare che l'utente possa utilizzare la bolla di tipo lista. & \textcolor{ForestGreen}{Implementato}\\
 \hline 
TSOpFu08-ls & Verificare che l'utente riceva una notifica al completamento di una lista. & \textcolor{Red}{Non Implementato}\\
 \hline 
TSOpFu01-mt & Verificare che l'utente possa utilizzare la bolla di tipo meteo. & \textcolor{Red}{Non Implementato}\\
 \hline 
TSObFu01-sd & Verificare che l'utente possa utilizzare la bolla di tipo sondaggio. & \textcolor{ForestGreen}{Implementato}\\
 \hline 
TSObFu04-sd & Verificare che il mittente e il ricevente visualizzino i risultati del sondaggio. & \textcolor{ForestGreen}{Implementato}\\
 \hline 
TSObFu07-sd & Verificare che l'utente non possa inviare la bolla nel caso in cui non siano stati inseriti il titolo e almeno due opzioni. & \textcolor{ForestGreen}{Implementato}\\
 \hline 
TSOpFu08-sd & Verificare che l'utente riceva una notifica quando tutti i partecipanti hanno espresso il voto. & \textcolor{Red}{Non Implementato}\\
 \hline 
TSOpFu01-tr & Verificare che l'utente possa utilizzare la bolla di tipo traduttore. & \textcolor{Red}{Non Implementato}\\
 \hline 
\end{longtable}
\captionof{table}{Test di sistema}
\end{center}

\subsection{Test di Validazione}

Tale tipologia di test viene utilizzata durante l'attività di collaudo
del prodotto finale, per accertare che il prodotto sia conforme alle
attese del committente. Per ogni test viene specificato il proprio
codice univoco, la descrizione, il requisito cui si fa riferimento, lo
stato di implementazione attuale e il risultato (Superato/Non
Superato).
I test non implementati sono relativi ai requisiti non obbligatori. 
 
\begin{center}
\begin{longtable}{|
*{1}{>{\centering\arraybackslash}p{2.8cm}|}
*{1}{>{\centering\arraybackslash}p{5cm}|}
*{1}{>{\centering\arraybackslash}p{2.5cm}|}
*{1}{>{\centering\arraybackslash}p{2.5cm}|}}
\hline \textbf{Test} & \textbf{Descrizione} & \textbf{Stato} & \textbf{Superato} \\
\hline \endhead

TVObDi01 & Verificare che l'SDK possa essere installata correttamente in un'istanza di Rocket.Chat. & \textcolor{ForestGreen}{Implementato} & \textcolor{ForestGreen}{Superato}\\
 \hline 
TVObDi02 & Verificare che l'SDK sia utilizzabile da alcune bolle predefinite. & \textcolor{ForestGreen}{Implementato} & \textcolor{ForestGreen}{Superato}\\
 \hline 
TVObDi03 & Verificare che sia incluso nell'SDK un set di API per lo sviluppo di bolle. & \textcolor{ForestGreen}{Implementato} & \textcolor{ForestGreen}{Superato}\\
 \hline 
TVObDi09 & Verificare che la demo sia installata su Heroku. & \textcolor{ForestGreen}{Implementato} & \textcolor{ForestGreen}{Superato}\\
 \hline 
TVObFu10 & Verificare che l'utente possa accedere alle bolle dall'interfaccia di Rocket.Chat. & \textcolor{ForestGreen}{Implementato} & \textcolor{ForestGreen}{Superato}\\
 \hline 
TVObFu10.2.1 & Verificare che l'utente possa visualizzare i tipi di bolla disponibili. & \textcolor{ForestGreen}{Implementato} & \textcolor{ForestGreen}{Superato}\\
 \hline 
TVObFu10.2.2 & Verificare che l'utente possa selezionare il tipo di bolla da inviare. & \textcolor{ForestGreen}{Implementato} & \textcolor{ForestGreen}{Superato}\\
 \hline 
TVObFu10.2.3 & Verificare che l'utente possa configurare la bolla tramite l'apposito menu. & \textcolor{ForestGreen}{Implementato} & \textcolor{ForestGreen}{Superato}\\
 \hline 
TVObFu10.2.4 & Verificare che l'utente possa inviare la bolla che ha configurato. & \textcolor{ForestGreen}{Implementato} & \textcolor{ForestGreen}{Superato}\\
 \hline 
TVObFu10.2.5 & Verificare che l'utente possa visualizzare lo storico delle bolle inviate. & \textcolor{ForestGreen}{Implementato} & \textcolor{ForestGreen}{Superato}\\
 \hline 
TVObFu10.3.1 & Verificare che l'utente possa visualizzare lo storico delle bolle ricevute. & \textcolor{ForestGreen}{Implementato} & \textcolor{ForestGreen}{Superato}\\
 \hline 
TVObFu11.1 & Verificare che lo sviluppatore possa creare le proprie bolle a partire dalla bolla vuota. & \textcolor{ForestGreen}{Implementato} & \textcolor{ForestGreen}{Superato}\\
 \hline 
TVObFu11.2.1.1 & Verificare che lo sviluppatore possa inserire un componente grafico. & \textcolor{ForestGreen}{Implementato} & \textcolor{ForestGreen}{Superato}\\
 \hline 
TVObFu11.2.1.2 & Verificare che lo sviluppatore possa inserire nella bolla un contenitore. & \textcolor{ForestGreen}{Implementato} & \textcolor{ForestGreen}{Superato}\\
 \hline 
TVObFu11.2.1.2.1 & Verificare che lo sviluppatore possa inserire nella bolla uno dei layout presenti nel SDK. & \textcolor{ForestGreen}{Implementato} & \textcolor{ForestGreen}{Superato}\\
 \hline 
TVObDi13 & Verificare che Monolith supporti i seguenti browser: 
\begin{itemize}
\item Google Chrome v59+
\item Firefox v54+
\item Internet Explorer v11+
\item Safari v10+
\end{itemize} & \textcolor{ForestGreen}{Implementato} & \textcolor{ForestGreen}{Superato}\\
 \hline 
TVObFu21 & Verificare che ogni client possa ricevere solo i dati che riguardano tutte le bolle presenti in rooms in cui sia presente anche l'utente. & \textcolor{ForestGreen}{Implementato} & \textcolor{ForestGreen}{Superato}\\
 \hline 
TVDeFu24 & Verificare che l'utente riceva notifiche per eventi di rilievo. & \textcolor{Red}{Non Implementato} & \textcolor{Red}{Non Superato}\\
 \hline 
TVObFu01-ls & Verificare che l'utente possa creare una lista con Checklist e inviarla all'interno di un canale Rocket.Chat. & \textcolor{ForestGreen}{Implementato} & \textcolor{ForestGreen}{Superato}\\
 \hline 
\end{longtable}
\captionof{table}{Test di validazione}
\end{center}

  \clearpage


  \appendix


  \section{Standard di qualità}

  Vengono riportati gli obbiettivi di qualità che il team Obelix
  dovrà raggiungere durante lo svolgimento del progetto.  Il team
  adotta  standard, modelli e metriche per verificare che ciascun
  obiettivo sia stato raggiunto.


  \subsection{Standard ISO/IEC 15504}


 La qualità di processo è definita dallo standard ISO/IEC 15504
 chiamato anche SPICE. 
 La qualità viene descritta in base a 6 possibili livelli di maturità
 ciascuno dei quali 
  %% Questo standard specifica come la qualità è collegata alla maturazione dei processi.
  %% Non sarebbe possibile definire la qualità del prodotto senza garantire la
  %% qualità del processo. La qualità del prodotto nasce dunque dalla qualità del
  %% processo. Per garantire tutto ciò il team Obelix fa uso dello standard ISO/IEC
  %% 15504 denominato SPICE, il quale fornisce gli strumenti necessari a valutare
  %% l’idoneità dei processi. Per l'esattezza sono previsti sei livelli di maturità:





  \begin{center}
    \includegraphics[scale=1.3]{img/Iso15504.jpg}
    \captionof{figure}{Standard ISO/IEC 15504 \\}
  \end{center}



  \begin{itemize}
  \item \textbf{Livello 0 - Incomplete process}\\
    Il processo non è implementato o non riesce a raggiungere i suoi obiettivi
  \item \textbf{Livello 1 - Performed process}\\
    Il processo viene messo in atto e raggiunge i suoi scopi
  \item  \textbf{Livello 2 - Managed process}\\
    Il processo viene eseguito sulla base di obiettivi ben definiti
  \item  \textbf{Livello 3 - Established process}\\
    Il processo viene eseguito in base ai principi dell'ingegneria del software
  \item  \textbf{Livello 4 - Predictable process}\\
    Il processo è attuato all'interno di limiti ben definiti
  \item  \textbf{Livello 5 - Optimizing process}\\
    Il processo è predicibile ed è in grado di adattarsi per raggiungere obiettivi specifici e rilevanti
  \end{itemize}

  Per valutare il livello di maturità di un processo si utilizzano le
  metriche descritte nella sezione \emph{2.8.1}, in particolare lo
  schedule variance aiuta a capire quando un processo raggiunge uno
  stato accettabile se il valore subisce al massimo lievi oscillazioni
  da quanto previsto.





  \subsection{Ciclo di Deming -  PDCA}
  Il ciclo di Deming (o PDCA) è un approccio manageriale per il
  controllo e il
  miglioramento continuo dei processi e dei prodotti ed è
  parte integrante della gestione della qualità.
  Il ciclo di Deming è suddiviso in quattro fasi: \\



  %\FloatBarrier
  \begin{center}
    % \centering
    \includegraphics[scale=0.3]{img/deming.png}
    \captionof{figure}{Ciclo di Deming}
  \end{center}
  %\FloatBarrier



  \begin{itemize}
  \item  \textbf{Plan} - stabilire gli obiettivi e i processi necessari per
fornire risultati in accordo con i risultati attesi, attraverso la
creazione di attese di produzione, di completezza e accuratezza delle
specifiche scelte. Quando possibile, avvio su piccola scala, per
verificare i possibili effetti.

    %% \begin{itemize}
    %% \item Precisazione degli obiettivi del miglioramento da attuare
    %% \item Raccolta dei dati relativi al problema o processo
    %% \item Mappatura del processo utilizzando un diagramma di flusso od altro strumento utile
    %% \item Individuazione delle cause principali dei vincoli da rimuovere
    %% \item Determinazione degli interventi necessari per risolvere la situazione
    %% \item Determinazione dei risultati attesi
    %% \item Definizione delle responsabilità per la fase di attuazione
    %% \item Pianificazione delle azioni da svolgere
    %% \item Pianificazione delle risorse
    %% \item Determinazione delle metriche per misurare i miglioramenti o gli scostamenti da quanto previsto
    %% \end{itemize}

  \item  \textbf{Do} -  Esecuzione del programma, dapprima in contesti
    circoscritti. Attuare il piano, eseguire il processo, creare il
    prodotto. Raccogliere i dati per la creazione di grafici e analisi
    da destinare alla fase di "Check" e "Act". 
    %% \begin{itemize}
    %% \item I responsabili individuati mettono in pratica le azioni previste
    %% \item Ogni soluzione è implementata per un periodo di prova
    %% \item Viene verificata l'adeguatezza delle soluzioni adottate rispetto agli obiettivi attesi
    %% \item Vengono formati i dipendenti sulle nuove modalità operative a fronte delle soluzioni adottate
    %% \end{itemize}

  \item  \textbf{Check} - Test e controllo, studio e raccolta dei
    risultati e dei riscontri. Studiare i risultati, misurati e
    raccolti nella fase del "Do" confrontandoli con i risultati
    attesi, obiettivi del "Plan", per verificarne le eventuali
    differenze. Cercare le deviazioni nell'attuazione del piano e
    focalizzarsi sulla sua adeguatezza e completezza per consentirne
    l'esecuzione. I grafici dei dati possono rendere questo molto più
    facile, in quanto è possibile vedere le tendenze di più cicli
    PDCA, convertendo i dati raccolti in informazioni. L'informazione
    è utile per realizzare il passo successivo : "Act".

  \item  \textbf{Act} - Azione per rendere definitivo e/o migliorare
    il processo (estendere quanto testato dapprima in contesti
    circoscritti all'intera organizzazione). Richiede azioni
    correttive sulle differenze significative tra i risultati
    effettivi e previsti. Analizza le differenze per determinarne le
    cause e dove applicare le modifiche per ottenere il miglioramento
    del processo o del prodotto. Quando un procedimento, attraverso
    questi quattro passaggi, non comporta la necessità di migliorare
    la portata a cui è applicato, il ciclo PDCA può essere raffinato
    per pianificare e migliorare con maggiore dettaglio la successiva
    iterazione, oppure l'attenzione deve essere posta in una diversa
    fase del processo. 

    %% \begin{itemize}
    %% \item Standardizzare il miglioramento ottenuto applicandolo in via definitiva
    %% \item Individuare eventuali esigenze di formazione del personale per rendere operative le soluzioni adottate
    %% \item Continuare a monitorare la situazione ripetendo il ciclo più volte fino a raggiungere i miglioramenti desiderati
    %% \item Individuare altre opportunità di miglioramento
    %% \end{itemize}
  \end{itemize}





















  \subsection{Standard ISO/IEC 9126}

  Aderendo a questo standard il team si impegna a garantire nel prodotto
  le seguenti qualità (definite con relativa metrica, misura e
  strumenti di controllo): \\


  \begin{center}
    \includegraphics[scale=0.5]{img/9126s.png}
    \captionof{figure}{Standard ISO/IEC 9126}
  \end{center}




  \begin{enumerate}
  \item \textbf{Funzionalità}: il sistema prodotto deve garantire tutte
    le funzionalità indicate nel documento \emph{\analisideirequisiti} L'implementazione dei requisiti deve essere più completa
    possibile

    \begin{itemize}
    \item \textbf{Misura}: l'unità di misura usata sarà la quantità di requisiti mappati sulle componenti del sistema create e funzionanti
    \item \textbf{Metrica}: la sufficienza è stabilita nel soddisfacimento di tutti i requisiti obbligatori
    \item \textbf{Strumenti}: per soddisfare questa qualità il sistema deve superare tutti i test previsti.
      Per informazioni dettagliate sugli strumenti si veda  \emph{\normediprogetto}
    \end{itemize}

  \item \textbf{Affidabilità}: il sistema deve dimostrarsi il più possibile robusto e di facile ripristino in caso di errori

    \begin{itemize}
    \item \textbf{Misura}: l’unità di misura utilizzata sarà la quantità di esecuzioni del sistema andate a buon fine
    \item \textbf{Metrica}: le esecuzioni dovranno spaziare il più
      possibile sulle varie parti del codice
    \item \textbf{Strumenti}: si vedano le metriche adottate per la
      copertura %%%%%%%%%%%%%%%%%%%%%%%%%%%%%%%METRICHE
    \end{itemize}

  \item \textbf{Usabilità}: il sistema prodotto deve risultare di facile
    utilizzo per la classe di utenti a cui è destinato. Tale sistema
    deve essere facilmente apprendibile e allo stesso tempo deve
    soddisfare tutte le necessità dell'utente
    \begin{itemize}
    \item \textbf{Misura}: l'unità di misura usata sarà una valutazione soggettiva dell'usabilità Questo è dovuto all'inesistenza di una metrica oggettiva adatta allo scopo
    \item \textbf{Metrica}:  non esiste una metrica adeguata che determinerà la sufficienza su questa qualità
    \item \textbf{Strumenti}: si vedano le  \normediprogetto
    \end{itemize}

  \item \textbf{Efficienza}: il sistema deve fornire tutte le funzionalità nel più breve tempo possibile,
    riducendo al minimo l'utilizzo di risorse
    \begin{itemize}
    \item \textbf{Misura}: il tempo che lo sviluppatore impiegherà per la creazione di una bolla interattiva
    \item \textbf{Metrica}: non è possibile definire una soglia oggettiva di sufficienza perché non è possibile valutare ogni possibile casistica di utilizzo
    \item \textbf{Strumenti}: si vedano le  \normediprogetto
    \end{itemize}

  \item \textbf{Manutenibilità}: il sistema deve essere comprensibile ed estensibile in modo facile e verificabile
    \begin{itemize}
    \item \textbf{Misura}: le metriche sul codice descritte nella sezione \emph{2.8.2}
      costituiscono l'unità di misura
    \item \textbf{Metrica}: il software avrà le caratteristiche di manutenibilità descritte. Questo sarà possibile se il prodotto avrà la sufficienza in tutte le metriche
    \item \textbf{Strumenti}: si vedano le  \normediprogetto
    \end{itemize}

  \item \textbf{Portabilità}: il sistema deve essere il più portabile possibile, in particolare dentro a Rocket.chat
    \begin{itemize}
    \item \textbf{Misura}: il front end deve rispettare gli standard W3C
    \item \textbf{Metrica}: il software avrà le caratteristiche di
      portabilità descritte. Questo sarà possibile se il prodotto
      avrà la sufficienza in tutte le metriche, descritte nella sezione \emph{2.8.2}
    \item \textbf{Strumenti}: si vedano le  \normediprogetto
    \end{itemize}
  \end{enumerate}




  \section{Resoconto delle attività di verifica}

  \subsection{Revisione dei Requisiti}

  L'attività di verifica svolta dai  \emph{Verificatori}  è avvenuta come determinato dal \emph{\pianodiprogetto} al termine della stesura di ogni documento previsto. La verifica svolta sui documenti e
  sui processi è avvenuta seguendo le indicazioni delle  \emph{\normediprogetto}  e misurando le
  metriche indicate nella sezione \emph{2.8.2}

  \subsubsection{Verifiche sui Processi}

  In questo periodo è stata svolta un' attività di walkthrough non avendo gli elementi
  per effettuare l'attività di inspection. Nella verifica dei
  documenti sono stati riscontrati soprattutto errori grammaticali e
  di battitura dovuti a disattenzioni durante la stesura. 

  \subsubsection{Documenti}
  Vengono qui riportati i valori dell'indice Gulpease per ogni documento durante l’analisi e relativo
  esito basato sui range stabiliti nella sezione \emph{2.8.2}
  \begin{center}
    \begin{tabular}{|c|c|c|}
      \hline
      \textbf{Documento} & \textbf{Valore indice} & \textbf{Esito} \\
      \hline
      \emph{Analisi dei Requisiti v1.0.0}  & 70 & superato \\
      % \hline
      %  \emph{Glossario v1.0.0}  & n & superato \\
      \hline
      \emph{Norme di Progetto v1.1.0}   & 58  & superato \\
      \hline
      \emph{Piano di Progetto v1.1.0}   & 65 & superato \\
      \hline
      \emph{Piano di Qualifica v1.0.0}   & 65 & superato \\
      \hline
      \emph{Studio di Fattibilità v1.0.0}  & 67 & superato \\
      \hline
    \end{tabular}
    \captionof{table}{Tabella risultati test Gulpease}
  \end{center}



  \subsection{Revisione di Progettazione}
  In questo periodo (antecedente la consegna di tale revisione) sono stati verificati i documenti ed i processi.

  \begin{itemize}
  \item Sono stati verificati i documenti applicando la procedura descritta nel documento \emph{Norme di progetto v3.0.0}
  \item \'E stata applicata \emph{l'analisi statica} secondo i criteri e le modalità indicata alla sezione \emph{2.7.1}.
 % \item Si è applicato il ciclo PDCA per rendere più efficiente ed efficace il processo di verifica.
  \item Si sono calcolate per i documenti le metriche descritte nella sezione \emph{2.8}.
  \item Il tracciamento (requisiti - componenti) è stato effettuato tramite un database MySQL creato ad uso interno.
  \item L'avanzamento dei processi è stato controllato e verificato secondo le metodiche descritte nelle \emph{Norme di Progetto v3.0.0}.

  \end{itemize}
  \subsubsection{Verifiche sui processi}

  Vengono qui riportati gli esiti delle verifiche sui processi produttivi.\\


  \textbf{Budget Variance / Schedule Variance}

  \begin{center}
    \begin{tabular}{|c|c|c|}

      \hline
      \textbf{Attività} & \textbf{SV} & \textbf{BV} \\
      \hline
      \emph{Analisi dei Requisiti 2.0.0} & 0€ & 0€ \\
      \hline
      \emph{Norme di Progetto 3.0.0} & +10€ & +70€ \\
      \hline
      \emph{Piano di Progetto 3.0.0} & 0€ & 0€ \\
      \hline
      \emph{Definizione di prodotto 1.0.0} & +25€ & -110€ \\
      \hline
      \emph{Piano di qualifica 3.0.0} & -10€ & 0€ \\
      \hline
      \emph{Glossario 2.0.0} & 0€ & 0€ \\
      \hline
    \end{tabular}
    \captionof{table}{Tabella risultati Budget Variance / Schedule Variance}
  \end{center}

  Complessivamente nel periodo di Revisione di Progettazione vengono misurate:
  \begin{itemize}
  \item Schedule Variance = 25€
  \item Budget Variance = -40€
  \end{itemize}

  Da valori di tali indici possiamo dedurre che:
  \begin{itemize}
  \item La data di fine della Revisione di Progettazione si è dimostrata essere un
    giorno prima di quella pianificata nel Piano di Progetto v3.0.0 .
    %% Aumentando il numero di ore di lavoro giornaliere ed utilizzando tutti gli slack
    %% inseriti durante la pianificazione il gruppo ha compresso i tempi con conseguente
    %% SV positivo;
  \item  Il limite inferiore di accettabilità del Budget Variance è pari a -436€.
    Il valore ottenuto risulta essere quindi accettabile. 
    %% Come già precisato, il
    %% gruppo per portare a termine gli obbiettivi entro le date imposte
    %% ha aumentato il numero di ore di lavoro giornaliere. 
    %% Questo ha quindi causato un aumento del BV in quanto le ore
    %% complessive di 
    %% lavoro sono state vicine a quelle preventivate, ma il periodo di lavoro è stato
    %% compresso.
  \end{itemize}


  \subsubsection{Verifica dei Documenti}
  Vengono qui riportati i valori dell'indice Gulpease per ogni documento durante l’analisi e relativo
  esito basato sui range stabiliti nella sezione \emph{2.8.2}
  \begin{center}
    \begin{tabular}{|c|c|c|}
      \hline
      \textbf{Documento} & \textbf{Valore indice} & \textbf{Esito} \\
      \hline
      \emph{Analisi dei Requisiti 2.0.0}  & 60.1 & superato \\
      \hline
      \emph{Norme di Progetto 3.0.0}   & 56.4  & superato \\
      \hline
      \emph{Piano di Progetto 3.0.0}   & 67.5 & superato \\
      \hline
      \emph{Piano di Qualifica 3.0.0}   & 61.4 & superato \\
      \hline
      \emph{Definizione di Prodotto 1.0.0}  & 76.9 & superato \\
      \hline
    \end{tabular}
    \captionof{table}{Tabella risultati test Gulpease}
  \end{center}
  
  
  %%%%%%%%%%%%%%%%%%%%%%%%%%%%%%%%%%%%%%%%%%%%%%%%%%%%%%%%%%%%%%%%%%%%%%%%%%%%%%%%%%%%%%%%%%%%%%%%55
  
  \subsection{Revisione di Qualifica}
  
    In questo periodo (antecedente la consegna di tale revisione) sono stati verificati i documenti ed i processi.
    
    \begin{itemize}
    	\item Sono stati verificati i documenti applicando la procedura descritta nel documento \emph{Norme di progetto v4.0.0}
    	\item \'E stata applicata \emph{l'analisi statica} secondo i criteri e le modalità indicata alla sezione \emph{2.7.1}.
    	% \item Si è applicato il ciclo PDCA per rendere più efficiente ed efficace il processo di verifica.
    	\item Si sono calcolate per i documenti le metriche descritte nella sezione \emph{2.8}.
   
    	\item L’avanzamento dei processi è stato controllato e verificato secondo le metodiche descritte nelle \emph{Norme di Progetto v4.0.0}.
    	
    \end{itemize}
    \subsubsection{Verifiche sui processi}
    
    Vengono qui riportati gli esiti delle verifiche sui processi produttivi.\\
    
    
    \textbf{Budget Variance / Schedule Variance}
     \begin{longtable}{|>{\centering}m{2cm}|c|c|c|c|c|}
     	\hline
     	\textbf{Metrica} & \textbf{Unità di misura} & \textbf{Risultato} & \textbf{Risultato massimo} & \textbf{Esito} & \textbf{Valore}\\
     	\hline
     	\endhead
     	\emph{Schedule Variance} & {Attività} & \textcolor{Green}{0} & / & Superato & Ottimale\\ \hline
     	\emph{Budget Variance} & {Euro} & \textcolor{Orange}{-687} & / & Non superato & /\\ \hline
    
    	\caption{RQ- BV e SV}\\
    \end{longtable}
    
    
    
    
    \begin{itemize}
    	\item La data di fine della Revisione di Progettazione si è dimostrata essere il
    	giorno stabilito nel Piano di Progetto v4.0.0 .
    	%% Aumentando il numero di ore di lavoro giornaliere ed utilizzando tutti gli slack
    	%% inseriti durante la pianificazione il gruppo ha compresso i tempi con conseguente
    	%% SV positivo;
    	\item  Il limite inferiore di accettabilità del Budget Variance è pari a -396€.
    	Il valore ottenuto risulta essere quindi non accettabile. 
    	%% Come già precisato, il
    	%% gruppo per portare a termine gli obbiettivi entro le date imposte
    	%% ha aumentato il numero di ore di lavoro giornaliere. 
    	%% Questo ha quindi causato un aumento del BV in quanto le ore
    	%% complessive di 
    	%% lavoro sono state vicine a quelle preventivate, ma il periodo di lavoro è stato
    	%% compresso.
    \end{itemize}
    
    \subsubsection{Produttività}
    
    \begin{longtable}{|>{\centering}m{2cm}|c|c|c|c|c|}
    	\hline
    	\textbf{Metrica} & \textbf{Unità di misura} & \textbf{Risultato} & \textbf{Risultato massimo} & \textbf{Esito} & \textbf{Valore}\\
    	\hline
    	\endhead
    	
    		\emph{Produttività di documentazione} & {n° parole/h} & \textcolor{Green}{116} & / & Superato & Ottimale \\ \hline
    	
   
    		
    		\emph{Produttività di codifica} & {n° linee codice/h} & \textcolor{Green}{86} & / & Superato & Accettabile \\ \hline
    		
   
	    	
    	  	\caption{RQ-Produttività}\\
    	  \end{longtable}
    	  
	
	
   \subsubsection{Verifica dei Documenti}
    Vengono qui riportati i valori dell'indice Gulpease per ogni documento durante l’analisi e relativo
    esito basato sui range stabiliti nella sezione \emph{2.8.2}
    \begin{center}
    	\begin{tabular}{|c|c|c|}
    		\hline
    		\textbf{Documento} & \textbf{Valore indice} & \textbf{Esito} \\
    		\hline
    		\emph{Analisi dei Requisiti 3.0.0}  & 66.7 & superato \\
    		\hline
    		\emph{Norme di Progetto 4.0.0}   & 55.3  & superato \\
    		\hline
    		\emph{Piano di Progetto 4.0.0}   & 66.3 & superato \\
    		\hline
    		\emph{Piano di Qualifica 4.0.0}   & 65.6 & superato \\
    		
    		\hline
    		\emph{Definizione di Prodotto 2.0.0}  & 69 & superato \\
    		\hline
    			\emph{Manuale Utente 1.0.0}  & 67.3 & superato \\
    			\hline
    	\end{tabular}
    	\captionof{table}{Tabella risultati test Gulpease}
    \end{center}
  
  \subsubsection{Metriche per software}
  \small{
  	Si noti che alcune componenti e i test a loro correlati non sono ancora state sviluppate, e che la loro assenza peserà nelle misurazioni.}
 
  
 
  \begin{longtable}{|>{\centering}m{2cm}|c|c|c|c|c|}
  	\hline
  	\textbf{Metrica} & \textbf{Unità di misura} & \textbf{Risultato} & \textbf{Risultato massimo} & \textbf{Esito} & \textbf{Valore}\\
  	\hline
  	\endhead
  	
  
  		\emph{Complessità Ciclomatica media} & {Cammini} & \textcolor{Green}{2.1} & / & Superato & Ottimale\\ \hline
  	\emph{Numero di metodi per package (max)} & {Metodi} & / & \textcolor{Green}{14} & Superato & Accettabile\\ \hline
  	\emph{Variabili non utilizzate e non definite} & {Variabili} & / & \textcolor{Green}{0} & Superato & Ottimale\\ \hline
  	\emph{Numero parametri per metodo (max)} & {Parametri} & / & \textcolor{Green}{5} & Superato & Accettabile\\ \hline
  	
  	
  
  	\emph{Halstead difficulty per-function (media) } & {indice complessità} & \textcolor{Green}{3.42} & / & Superato & Ottimale\\ \hline
  	\emph{Halstead volume per-function (media)} & {indice complessità} & \textcolor{Green}{100.76} & / & Superato & Ottimale\\ \hline
  	\emph{Halstead effort per-function} & {indice complessità} & \textcolor{Green}{344.59} & / & Superato & Accettabile\\ \hline
  	
  	
  	\emph{Maintainability index} & {Valore} & \textcolor{Green}{71.98} & / & Superato  & Accettabile \\ \hline
  	
  	\emph{Statement Coverage} & {Percentuale} & \textcolor{Green}{77.4\%} & / & Superato & Accettabile\\ \hline
  		
  		
  	\emph{Branch Coverage} & {Percentuale} & \textcolor{Green}{83\%} & / & Superato & Accettabile\\ \hline
  	
  	
  	\emph{Percentuale superamento test} & {Percentuale} & \textcolor{Green}{82\%} & / & Superato & Accettabile\\ \hline
  	
  	 
  	\caption{RQ-Metriche per Software}\\
  \end{longtable}

  \subsection{Revisione di Accettazione}
  
  In questo periodo (antecedente la consegna di tale revisione) sono stati verificati i documenti ed i processi.
  
  \begin{itemize}
  	\item Sono stati verificati i documenti applicando la procedura descritta nel documento \emph{Norme di progetto v5.0.0}
  	\item \'E stata applicata \emph{l'analisi statica} secondo i criteri e le modalità indicata alla sezione \emph{2.7.1}.
  	% \item Si è applicato il ciclo PDCA per rendere più efficiente ed efficace il processo di verifica.
  	\item Si sono calcolate per i documenti le metriche descritte nella sezione \emph{2.8}.
  	
  	\item L’avanzamento dei processi è stato controllato e verificato secondo le metodiche descritte nelle \emph{Norme di Progetto v5.0.0}.
  	
  \end{itemize}
  \subsubsection{Verifiche sui processi}
  
  Vengono qui riportati gli esiti delle verifiche sui processi produttivi.\\
  
  
  \textbf{Budget Variance / Schedule Variance}
  \begin{longtable}{|>{\centering}m{2cm}|c|c|c|c|c|}
  	\hline
  	\textbf{Metrica} & \textbf{Unità di misura} & \textbf{Risultato} & \textbf{Risultato massimo} & \textbf{Esito} & \textbf{Valore}\\
  	\hline
  	\endhead
  	\emph{Schedule Variance} & {Attività} & \textcolor{Green}{0} & / & Superato & Ottimale\\ \hline
  	\emph{Budget Variance} & {Euro} & \textcolor{Orange}{-94} & / & Non superato & /\\ \hline
  	
  	\caption{RA- BV e SV}\\
  \end{longtable}
  
  
  
  
  \begin{itemize}
  	\item La data di fine della Revisione di Progettazione si è dimostrata essere il
  	giorno stabilito nel Piano di Progetto v4.0.0 .
  	%% Aumentando il numero di ore di lavoro giornaliere ed utilizzando tutti gli slack
  	%% inseriti durante la pianificazione il gruppo ha compresso i tempi con conseguente
  	%% SV positivo;
  	\item  Il limite inferiore di accettabilità del Budget Variance è pari a 0€.
  	Il valore ottenuto risulta essere quindi non accettabile. 
  	%% Come già precisato, il
  	%% gruppo per portare a termine gli obbiettivi entro le date imposte
  	%% ha aumentato il numero di ore di lavoro giornaliere. 
  	%% Questo ha quindi causato un aumento del BV in quanto le ore
  	%% complessive di 
  	%% lavoro sono state vicine a quelle preventivate, ma il periodo di lavoro è stato
  	%% compresso.
  \end{itemize}
  
  \subsubsection{Produttività}
  
  \begin{longtable}{|>{\centering}m{2cm}|c|c|c|c|c|}
  	\hline
  	\textbf{Metrica} & \textbf{Unità di misura} & \textbf{Risultato} & \textbf{Risultato massimo} & \textbf{Esito} & \textbf{Valore}\\
  	\hline
  	\endhead
  	
  	\emph{Produttività di documentazione} & {n° parole/h} & \textcolor{Green}{106} & / & Superato & Ottimale \\ \hline
  	
  	
  	
  	\emph{Produttività di codifica} & {n° linee codice/h} & \textcolor{Green}{70} & / & Superato & Accettabile \\ \hline
  	
  	
  	
  	
  	\caption{RA-Produttività}\\
  \end{longtable}
  
  
  
  
  
  
  
  
  \subsubsection{Verifica dei Documenti}
  Vengono qui riportati i valori dell'indice Gulpease per ogni documento durante l’analisi e relativo
  esito basato sui range stabiliti nella sezione \emph{2.8.2}
  \begin{center}
  	\begin{tabular}{|c|c|c|}
  		\hline
  		\textbf{Documento} & \textbf{Valore indice} & \textbf{Esito} \\
  		\hline
  		\emph{Analisi dei Requisiti 4.0.0}  & 66.7 & superato \\
  		\hline
  		\emph{Norme di Progetto 5.0.0}   & 55.3  & superato \\
  		\hline
  		\emph{Piano di Progetto 5.0.0}   & 66.3 & superato \\
  		\hline
  		\emph{Piano di Qualifica 5.0.0}   & 65.6 & superato \\
  		
  		\hline
  		\emph{Definizione di Prodotto 3.0.0}  & 69 & superato \\
  		\hline
  		\emph{Manuale Utente 2.0.0}  & 67.3 & superato \\
  		\hline
  	\end{tabular}
  	\captionof{table}{Tabella risultati test Gulpease}
  \end{center}
  
  \subsubsection{Metriche per software}
  
  
  
  
  \begin{longtable}{|>{\centering}m{2cm}|c|c|c|c|c|}
  	\hline
  	\textbf{Metrica} & \textbf{Unità di misura} & \textbf{Risultato} & \textbf{Risultato massimo} & \textbf{Esito} & \textbf{Valore}\\
  	\hline
  	\endhead
  	
  	
  	\emph{Complessità Ciclomatica media} & {Cammini} & \textcolor{Green}{2.3} & / & Superato & Ottimale\\ \hline
  	\emph{Numero di metodi per package (max)} & {Metodi} & / & \textcolor{Green}{14} & Superato & Accettabile\\ \hline
  	\emph{Variabili non utilizzate e non definite} & {Variabili} & / & \textcolor{Green}{0} & Superato & Ottimale\\ \hline
  	\emph{Numero parametri per metodo (max)} & {Parametri} & / & \textcolor{Green}{5} & Superato & Accettabile\\ \hline
  	
  	
  	
  	\emph{Halstead difficulty per-function (media) } & {indice complessità} & \textcolor{Green}{3.62} & / & Superato & Ottimale\\ \hline
  	\emph{Halstead volume per-function (media)} & {indice complessità} & \textcolor{Green}{100.82} & / & Superato & Ottimale\\ \hline
  	\emph{Halstead effort per-function} & {indice complessità} & \textcolor{Green}{364.9684} & / & Superato & Accettabile\\ \hline
  	
  	
  	\emph{Maintainability index} & {Valore} & \textcolor{Green}{73.86} & / & Superato  & Accettabile \\ \hline
  	
  	\emph{Statement Coverage} & {Percentuale} & \textcolor{Green}{86.2\%} & / & Superato & Ottimale\\ \hline
  	
  	
  	\emph{Branch Coverage} & {Percentuale} & \textcolor{Green}{88.3\%} & / & Superato & Ottimale\\ \hline
  	
  	
  	\emph{Percentuale superamento test} & {Percentuale} & \textcolor{Green}{100\%} & / & Superato & Ottimale\\ \hline
  	
  	
  	\caption{RA-Metriche per Software}\\
  \end{longtable}

\subsection{Riepilogo obiettivi raggiunti}

Di seguito viene riportata una tabella riassuntiva, in merito al raggiungimento degli obiettivi fissati.

\begin{center}
	\centering
	\begin{tabular}{|c|c|c|}
		\hline
		\textbf{ID} & \textbf{Nome} & \textbf{Raggiunto SI/NO} \\
		\hline OQD001 & Leggibilità dei documenti & SI  \\
		\hline OQD002 & Correttezza ortografica dei documenti & SI \\
		\hline OQD003 & Produttività nella stesura dei documenti & SI \\
		\hline OQS001 & Produttività nella stesura del codice & SI \\
		\hline OQS002 & Copertura del codice & SI  \\
		\hline OQS003 & Copertura dei test & SI \\
		\hline OQS004 & Manutenzione e comprensibilità del codice & SI  \\
	
		\hline
	\end{tabular}
\end{center}

\section{Valutazioni critiche finali}

L'obiettivo di questa appendice è quello di valutare lo svolgimento del progetto per mettere in evidenza i problemi organizzativi emersi durante l'esecuzione dei lavori al fine di proporre soluzioni utili alla loro risoluzione. A questo punto del progetto, la valutazione non può risultare utile per il miglioramento di quest'ultimo, ma potrebbe esserlo per progetti futuri. \\

Il principale problema riscontrato riguarda l'organizzazione del gruppo. Soprattutto nel primo periodo, la comunicazione tra i membri è risultata complicata e non è stato sempre possibile organizzare incontri in modo che tutti i membri fossero presenti. 

Per migliorare l'organizzazione è utile che vengano utilizzati gli strumenti adatti (Slack, Telegram, Skype) in modo più adeguato e che ogni membro del gruppo garantisca la propria disponibilità in modo più realistico. \\
 
Altre difficoltà sono state riscontrate nella pianificazione delle attività. Il gruppo non è riuscito a rispettare le scadenze interne programmate causando più volte la modifica della pianificazione. 
Inoltre una sottovalutazione dei tempi nei primi periodi non rendicontati ha provocato a cascata un ritardo nei periodi successivi. Non siamo stati in grado di correggere questo problema a causa di una sottovalutazione dello stesso e per le grandi quantità di tempo spese a risolvere problemi tecnici relativi agli strumenti.

Per poter essere rispettata una pianificazione dovrebbe prevedere una suddivisione delle attività che permetta l'assegnazione di ognuna di esse ad una singola persona, il parallelismo dove possibile e la non sovrapposizione di attività critiche.   \\

Altri problemi riscontrati riguardano la decomposizione e classificazione dei requisiti. Inizialmente l'analisi effettuata non teneva conto di tutti i requisiti necessari ed era limitata ai soli requisiti espliciti. Durante la prima stesura dell'analisi dei requisiti non era completamente chiaro quale fosse lo scopo del processo di Requirement Elicitation. Questo ha portato a requisiti parziali e limitati ai requisiti utente. 

Per effettuare un'analisi completa che tenga conto sia dei requisiti espliciti che di quelli impliciti è necessario studiare, in modo approfondito, il dominio del problema, intervistare il proponente e i possibili utenti del prodotto da realizzare. \\
 
Inoltre, per quanto riguarda l'utilizzo delle tecnologie e degli strumenti scelti sono emerse diverse difficoltà che non hanno reso semplice le attività di progettazione e verifica. 
Non siamo stati in grado di assimilare sufficientemente le tecnologie e gli strumenti inizialmente sconosciuti con cui era richiesto lavorare. Questo ha introdotto problemi nel completamento dei requisiti e in tutte le fasi della progettazione. Alcune di queste difficoltà hanno avuto strascichi fino alle prime versioni del codice. 

L'ovvia soluzione è richiedere a sé stessi un maggior livello di confidenza con strumenti e tecnologie prima di utilizzarli in ambito di produzione. \\

Nel complesso l'insieme di tutti i fattori illustrati ha precluso la realizzazione di un modello di sviluppo veramente incrementale. Le ripetute iterazioni attraverso i periodi hanno aumentato la confusione e sottratto tempo che poteva essere impiegato per migliorare. 
